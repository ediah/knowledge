\documentclass[a4paper,11pt,openany]{book}
\usepackage{knowledge}

\title{Математические основы криптологии}
\author{Автор курса: Применко Эдуард Андреевич \\ 
		Составитель: Смирнов Дмитрий Константинович}
\date{Версия от \currenttime, \today}

\begin{document}

\maketitle
\tableofcontents

\mainmatter
\chapter{Домашние задания}

\section{Элементы теории групп}

Задачи в этом разделе решаются со следующими параметрами:

\medskip

{\centering
\begin{tabular}{||c|c|c||}
\hline
\textbf{p} & \textbf{g} & \textbf{k} \\
\hline
23 & -8 & 22 \\
\hline
\end{tabular}

}

\medskip

\task{Убедиться, что $g \in \mathbb{Z}_p^*$ -- примитивный элемент $\mathbb{Z}_p$.}

Так как $p = 23$ -- простое число, то $\phi(p) = p - 1 = 22$. Разложим это число на простые множители: $\phi(p) = 2 \cdot 11$. Тогда достаточно проверить следующие 2 неравенства:
$$ g^{ \frac{\phi(p)}{2} } = (-8) ^ {11} = 15 \cdot 15 ^ {10} = 15 \cdot 18 ^ 5 = 17 \cdot 2 ^ 2 = 22 \not\equiv 1 \pmod p,$$
$$ g^{ \frac{\phi(p)}{11} } = (-8) ^ {2} = 18 \not\equiv 1 \pmod p,$$

Делаем вывод, что $g$ действительно является примитивным элементом $\mathbb{Z}_p$.

\task{Найти образующий элемент $h$ группы $\mathbb{Z}_{p^2}^*$}

Образующий элемент группы $\mathbb{Z}_{p^n}^*, n \ge 2$ имеет вид:
$$ h = g + t_0 p, \; t_0 \not \equiv g \nu \pmod p; \; \nu = ( \frac{ g ^ {\frac{p -1}{2}} + 1}{ p } )(\!\!\!\!\!\! \mod p) \cdot (-2)(\!\!\!\!\!\! \mod p)$$
\noindent Таким образом,
$$ \nu = ( \frac{ (-8) ^ {\frac{23 - 1}{2}} + 1}{ 23 } )(\!\!\!\!\!\! \mod 23) \cdot (-2)(\!\!\!\!\!\! \mod 23) = ( 1 \cdot (-2))(\!\!\!\!\!\! \mod 23) = 21$$
$$ t_0 \not \equiv (-8) \cdot 21 \!\!\!\pmod {23} = 16 \!\!\!\pmod {23} $$
$$ t_1 = 1 \Rightarrow h = (-8) + 1 * 23 = 15 $$
Следовательно, $h = 15$ -- образующий элемент группы $\mathbb{Z}_{23^2}^*$

\task{Подсчитать число образующих группы $\mathbb{Z}_{p^3}^*$}

Число образующих группы $\mathbb{Z}_{23^3}^*$ равно $\phi(23^3) = (23 - 1) 23 ^ {3 - 1} = 11638.$

\task{Найти элемент $a$ группы $\mathbb{Z}_{p^2}^*$ порядка $k$}

Так как $\forall$ натурального $k>1$ и простого $p \ge 3$ группа $\mathbb{Z}_{p^k}^*$ является циклической, то $\mathbb{Z}_{23^2}^*$ -- циклическая группа. Элемент порядка $k$ в циклической группе порядка $N$ имеет вид $h^r,$ где $r = \frac{N}{k}.$ Таким образом,

$$ a = h ^ { \frac{\phi(p ^ 2)}{k} } = 15 ^ { \frac{22 * 23}{22} } = 15 ^ {23} = 130 $$

\task{Решить сравнение $a^x \equiv b \pmod p$}

\medskip

{\centering
\begin{tabular}{||c|c|c||}
\hline
\textbf{p} & \textbf{a} & \textbf{b} \\
\hline
701 & 2 & 163 \\
\hline
\end{tabular}

}

\medskip

\noindent \textbf{I. Алгоритм согласования }

1. Убедимся в том, что $a=2$ -- примитивный элемент группы $\mathbb{Z}_{701}$.
$$\phi(701) = 700 = 2^2 \cdot 5^2 \cdot 7$$
$$ g^{ \frac{\phi(p)}{2} } = 2 ^ {350} = 700 \not\equiv 1 \pmod p,$$
$$ g^{ \frac{\phi(p)}{5} } = 2 ^ {140} = 210 \not\equiv 1 \pmod p,$$
$$ g^{ \frac{\phi(p)}{7} } = 2 ^ {100} = 19 \not\equiv 1 \pmod p,$$
$$ g^{ \phi(p) } = 2 ^ {700} = 1 \equiv 1 \pmod p,$$
\noindent Таким образом, порядок элемента $a$ равен $ord(a) = 700$.

2. Выбираем минимальное $m \colon m^2 \ge ord(a) \Rightarrow m = 27.$ 

3. Вычисляем $c = a^m = 2 ^ {27} = 62.$

4. Составляем два множества:

\medskip

{\centering
\begin{tabular}{||c|c|c|c|c|c|c|c|c|c|c|c|c|c|c|c|c|c|c|c|c|c|c|c|c|c|c||}
\hline
$i$ & 1 & 2 & 3 & 4 & 5 & 6 & 7 & 8 & 9 & 10 & 11 & 12 & 13 & 14 \\
\hline
$c^i$ & 62 & 339 & 689 & 658 & 138 & 144 & 516 & 447 & 375 & 117 & 244 & 407 & 699 & 577 \\
\hline
\end{tabular}

}

\medskip

{\centering
\begin{tabular}{||c|c|c|c|c|c|c|c|c|c|c|c|c|c|c|c|c|c|c|c|c|c|c|c|c|c||}
\hline
$i$ & 15 & 16 & 17 & 18 & 19 & 20 & 21 & 22 & 23 & 24 & 25 & 26 & 27 \\
\hline
$c^i$ & 23 & 24 & 86 & 425 & 413 & 370 & 508 & 652 & 467 & 213 & 588 & 4 & 248 \\
\hline
\end{tabular}

}

\medskip

{\centering
\begin{tabular}{||c|c|c|c|c|c|c|c|c|c|c|c|c|c|c|c|c|c|c|c|c|c|c|c|c|c|c||}
\hline
$j$ & 0 & 1 & 2 & 3 & 4 & 5 & 6 & 7 & 8 & 9 & 10 & 11 & 12 & 13 \\
\hline
$ba^j$ & 163 & 326 & 652 & 603 & 505 & 309 & 618 & 535 & 369 & 37 & 74 & 148 & 296 & 592 \\
\hline
\end{tabular}

}

\medskip

{\centering
\begin{tabular}{||c|c|c|c|c|c|c|c|c|c|c|c|c|c|c|c|c|c|c|c|c|c|c|c|c|c||}
\hline
$j$ & 14 & 15 & 16 & 17 & 18 & 19 & 20 & 21 & 22 & 23 & 24 & 25 & 26 \\
\hline
$ba^j$ & 483 & 265 & 530 & 359 & 17 & 34 & 68 & 136 & 272 & 544 & 387 & 73 & 146 \\
\hline
\end{tabular}

}

\medskip

\noindent В таблицах совпадают элементы под номерами $i = 22$ и $j = 2.$

5. Таким образом, $x = mi - j = 27 \cdot 22 - 2 = 592.$ 

\noindent Ответ: $x = 592$.

\newpage

\noindent \textbf{II. Алгоритм Полига-Хеллмана}

Порядок поля $\mathbb{Z}_{701}$ равен $N = \phi(701) = 700 = 2 ^ 2 \cdot 5 ^ 2 \cdot 7$. Количество простых множителей в разложении этого числа $t = 3$.

1. Вычисляем матрицу с элементами $(i, j) = a ^ { j  \frac{N}{p_i}} $, $i = \overline{1, t},\ j =~\overline{0, p_i - 1}$:

\medskip

{\centering
\begin{tabular}{||c|c|c|c|c|c|c|c||}
\hline
\diagbox{$p_i$}{$j$} & 0 & 1 & 2 & 3 & 4 & 5 & 6 \\
\hline

2 & $2 ^ {0 \cdot \frac{700}{2} }$ & $2 ^ {1 \cdot \frac{700}{2} }$ & - & - & - & - & -\\

\hline
5 & $2 ^ {0 \cdot \frac{700}{5} }$ & $2 ^ {1 \cdot \frac{700}{5} }$ & $2 ^ {2 \cdot \frac{700}{5} }$ & $2 ^ {3 \cdot \frac{700}{5} }$ & $2 ^ {4 \cdot \frac{700}{5} }$ & - & -\\
\hline
7 & $2 ^ {0 \cdot \frac{700}{7} }$ & $2 ^ {1 \cdot \frac{700}{7} }$ & $2 ^ {2 \cdot \frac{700}{7} }$ & $2 ^ {3 \cdot \frac{700}{7} }$ & $2 ^ {4 \cdot \frac{700}{7} }$ & $2 ^ {5 \cdot \frac{700}{7} }$ & $2 ^ {6 \cdot \frac{700}{7} }$ \\
\hline
\end{tabular}

}

\medskip

{\centering
\begin{tabular}{||c|c|c|c|c|c|c|c||}
\hline
\diagbox{$p_i$}{$j$} & 0 & 1 & 2 & 3 & 4 & 5 & 6 \\
\hline

2 & 1 & 700 & - & - & - & - & -\\

\hline
5 & 1 & 210 & 638 & 89 & 464 & - & -\\
\hline
7 & 1 & 19 & 361 & 550 & 636 & 167 & 369 \\
\hline
\end{tabular}

}

\medskip

2. Далее находим  $x_i = \log_a b \pmod { p_i^{k_i} } = \gamma_0 + \gamma_1 p_i + ... + \gamma_{k_i - 1} p_i ^ {k_i - 1}, \gamma_j \in~\mathbb{Z}_p$. Последовательно находим $\gamma_j$ из $ M(p, \gamma_{j}) = b_j ^ { \frac{N}{p ^ {j + 1}} }$, где $b_j = ba^{-\gamma_0 - \gamma_1 p - ... - \gamma_{j - 1} p ^ {j - 1}}$, а $M$ -- определённая выше матрица.

а) $x_1 = \log_2 163 \pmod { 2^2 }, \ p = 2, \ k = 2$

\noindent $M(p, \gamma_0) = b ^ { \frac{N}{p} } = 163 ^ { \frac{700}{2} } = 1 \Rightarrow \gamma_0 = 0, \ b_1 = ba^{-\gamma_0} = 163 \cdot 2 ^ {-0} = 163$

\noindent $M(p, \gamma_1) = b_1 ^ { \frac{N}{p^2} } = 163 ^ { \frac{700}{4} } = 1 \Rightarrow \gamma_1 = 0$

\noindent $\Rightarrow x_1 = \gamma_0 + \gamma_1 p = 0 + 0 \cdot 2 = 0$

б) $x_2 = \log_2 163 \pmod { 5^2 }, \ p = 5, \ k = 2$

\noindent $M(p, \gamma_0) = b ^ { \frac{N}{p} } = 163 ^ { \frac{700}{5} } = 638 \Rightarrow \gamma_0 = 2, \ b_1 = ba^{-\gamma_0} = 163 \cdot 2 ^ {-2} = 216$

\noindent $M(p, \gamma_1) = b_1 ^ { \frac{N}{p^2} } = 216 ^ { \frac{700}{25} } = 89 \Rightarrow \gamma_1 = 3$

\noindent $\Rightarrow x_2 = \gamma_0 + \gamma_1 p = 2 + 3 \cdot 5 = 17$

в) $x_3 = \log_2 163 \pmod { 7 }, \ p = 7, \ k = 1$

\noindent $M(p, \gamma_0) = b ^ { \frac{N}{p} } = 163 ^ { \frac{700}{7} } = 636 \Rightarrow \gamma_0 = 4$

\noindent $\Rightarrow x_3 = \gamma_0 = 4$

3. На основе вычисленных выше значений $x_1, x_2, ..., x_t$ и китайской теоремы об остатках находим искомый логарифм:

$$x = \sum x_i \frac{N}{p_i^{k_i}} [ ( \frac{N}{p_i^{k_i}} ) ^ {-1} \!\!\!\!\pmod {p_i^{k_i}} ] \!\!\!\!\pmod N = 0 \cdot \frac{700}{2^2} [ (\frac{700}{2^2}) ^ {-1} \!\!\!\!\pmod {2^2}] + $$ 
$$+ 17 \cdot \frac{700}{5^2} [ (\frac{700}{5^2}) ^ {-1} \!\!\!\!\pmod {5^2}] + 4 \cdot \frac{700}{7} [ (\frac{700}{7}) ^ {-1} \!\!\!\!\pmod 7]  \pmod {700} = $$
$$ = 476 \cdot [ 28 ^ {-1} \!\!\!\!\pmod {25}] + 400 \cdot [100 ^ {-1} \!\!\!\!\pmod 7] \pmod {700} = $$
$$ = 476 \cdot 17 + 400 \cdot 4 \pmod {700} = 592$$

\noindent Ответ: $x = 592$.

\section{}

\chapter{Билеты}
\section{Делимость в кольце целых чисел. НОД, алгоритм Евклида. Критерий взаимной простоты двух чисел.}

\theorem{(о делении с остатком) Пусть $a>0$ и $b>0$ -- целые числа. Тогда $a$ единственным образом представимо в виде 

$$ a = bq + r, \ 0 \le r < b. $$

\noindent Число $q$ -- \textbf{неполное частное}

}

\section{Сравнения и их свойства. Китайская теорема об остатках. Кольцо вычетов. Функция Эйлера и её свойства.}
\section{Теоремы Эйлера и Ферма. Критерий обратимости, алгоритм вычисления обратного элемента.}
\section{Криптографическая теорема (обоснование криптосистемы РСА).}
\section{Теорема о цикличности мультипликативной группы по примарному модулю.}
\section{Решение сравнений первой степени.}
\section{Сравнения второй степени. Символ Лежандра и его свойства.}
\section{Алгоритмы решения сравнений второй степени по простому модулю.}
\section{Символ Якоби и его свойства. Числа Блюма и их свойства. Эквивалентность задачи факторизации и решения сравнения второй степени.}
\section{Алгоритмы решения сравнений второй степени по примарному и составному модулю.}
\section{Группа, порядок элемента. Теорема Лагранжа.}
.
\section{Нормальный делитель, фактор – группа, первая теорема о гомоморфизме.}
\section{Кольцо многочленов, идеал, теорема Безу, кольцо главных идеалов.}
\section{Конечное поле. Теорема о простом подполе конечного поля. Строение конечного поля. Теорема о примитивном элементе.}
\section{Построение конечных полей. Алгоритм вычисления обратного элемента. Арифметические операции в конечном поле.}
\section{Алгоритмы вычисления дискретного алгоритма.}
\section{Криптосистема Эль - Гамаля. Протокл Диффи - Хеллмана.}
\section{Минимальный многочлен и его свойства. Теорема об изоморфизме конечных полей одной мощности.}
\section{Примитивный многочлен и его свойства. Теорема о разложении многочлена $f(x) = x p^n – x$ на неприводимые многочлены. Критерий принадлежности элемента поля собственному подполю.}
\section{Теорема о группе автоморфизмов конечного поля.}
\section{Рекуррентные последовательности над конечным полем, линейные рекуррентные последовательности (ЛРП). Характеристический и минимальный многочлен ЛРП и их свойства.}
\section{Теорема об определении структуры ЛРП по её характеристическому многочлену. Теорема о ЛРП максимального периода.}
\section{Прямое произведение групп. Теорема о представлении группы в виде прямого произведения своих подгрупп.}
\section{Теорема о примарной абелевой группе.}
.
\section{Теорема о разложении конечной абелевой группы в произведение своих циклических подгрупп.}
\section{Нормализатор, централизатор, класс сопряженных элементов конечной группы. Теорема о числе множеств сопряженных с данным. Теорема о центре примарной группы. Теорема Коши.}
\section{Двойные смежные классы и их свойства. Теорема Силова (первая)}
\section{Вторая и третья теоремы Силова.}
\section{Группы подстановок. Инвариантное множество, орбита. Теорема об индексе стабилизатора группы. Теорема о транзитвности нормализатора подгруппы транзитвной группы. (Ут . 13.4).}
\section{Лемма Бернсайда.}
\section{Регулярные и полурегулярные группы. Порядок полурегулярной группы.}
\section{Блоки и импримитивные группы. Критерий импримитивности. Теорема о импримитивности транзитивной группы с интранзитивным нормальным делителем.}
\section{Примитивные группы. Кратная транзитивность. Критерий кратной транзитивности.}
\section{Теорема о группе автоморфизмов конечной группы.}
\section{Утверждение об изоморфизме стабилизатора и специальной группы автоморфизмов регулярной подгруппы (Ут . 13.5). Утверждение о порядке регулярного нормального делителя кратно транзитивной группы.}
\section{Простая группа. Теорема о простоте знакопеременной группы. Теорема о нормальном делителе симметрической группы.}

\end{document}