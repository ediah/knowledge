\documentclass[a4paper,11pt,openany]{book}
\usepackage{knowledge}

\title{Математические основы криптологии}
\author{Автор курса: Применко Эдуард Андреевич \\ 
		Составитель: Смирнов Дмитрий Константинович}
\date{2022 г.}

\begin{document}

\maketitle
\tableofcontents

\mainmatter
\chapter{Домашние задания}

\section{Элементы теории групп}

Задачи в этом разделе решаются со следующими параметрами:

\medskip

{\centering
\begin{tabular}{||c|c|c||}
\hline
\textbf{p} & \textbf{g} & \textbf{k} \\
\hline
23 & -8 & 22 \\
\hline
\end{tabular}

}

\medskip

\task{Убедиться, что $g \in \mathbb{Z}_p^*$ -- примитивный элемент $\mathbb{Z}_p$.}

Так как $p = 23$ -- простое число, то $\phi(p) = p - 1 = 22$. Разложим это число на простые множители: $\phi(p) = 2 \cdot 11$. Тогда достаточно проверить следующие 2 неравенства:
$$ g^{ \frac{\phi(p)}{2} } = (-8) ^ {11} = 15 \cdot 15 ^ {10} = 15 \cdot 18 ^ 5 = 17 \cdot 2 ^ 2 = 22 \not\equiv 1 \pmod p,$$
$$ g^{ \frac{\phi(p)}{11} } = (-8) ^ {2} = 18 \not\equiv 1 \pmod p,$$

\noindent и одно равенство:
$$ g ^ {\phi (p)} = (-8) ^ {22} = 18 ^ {11} = 18 \cdot 2 ^ 5 = 18 \cdot 9 \equiv 1 \pmod p.$$

Делаем вывод, что $g$ действительно является примитивным элементом $\mathbb{Z}_p$.

\task{Найти образующий элемент $h$ группы $\mathbb{Z}_{p^2}^*$}

Образующий элемент группы $\mathbb{Z}_{p^n}^*, n \ge 2$ имеет вид:
$$ h = g + t_0 p, \; t_0 \not \equiv g \nu \pmod p; \; \nu = ( \frac{ g ^ {\frac{p -1}{2}} + 1}{ p } )(\!\!\!\!\!\! \mod p) \cdot (-2)(\!\!\!\!\!\! \mod p)$$
\noindent Таким образом,
$$ \nu = ( \frac{ (-8) ^ {\frac{23 - 1}{2}} + 1}{ 23 } )(\!\!\!\!\!\! \mod 23) \cdot (-2)(\!\!\!\!\!\! \mod 23) = ( 1 \cdot (-2))(\!\!\!\!\!\! \mod 23) = 21$$
$$ t_0 \not \equiv (-8) \cdot 21 \!\!\!\pmod {23} = 16 \!\!\!\pmod {23} $$
$$ t_1 = 1 \Rightarrow h = (-8) + 1 * 23 = 15 $$
Следовательно, $h = 15$ -- образующий элемент группы $\mathbb{Z}_{23^2}^*$

\task{Подсчитать число образующих группы $\mathbb{Z}_{p^3}^*$}



\task{Найти элемент $a$ группы $\mathbb{Z}_{p^2}^*$ порядка $k$}


\chapter{Билеты}
\section{Делимость в кольце целых чисел. НОД, алгоритм Евклида. Критерий взаимной простоты двух чисел.}
\section{Сравнения и их свойства. Китайская теорема об остатках. Кольцо вычетов. Функция Эйлера и её свойства.}
\section{Теоремы Эйлера и Ферма. Критерий обратимости, алгоритм вычисления обратного элемента.}
\section{Криптографическая теорема (обоснование криптосистемы РСА).}
\section{Теорема о цикличности мультипликативной группы по примарному модулю.}
\section{Решение сравнений первой степени.}
\section{Сравнения второй степени. Символ Лежандра и его свойства.}
\section{Алгоритмы решения сравнений второй степени по простому модулю.}
\section{Символ Якоби и его свойства. Числа Блюма и их свойства. Эквивалентность задачи факторизации и решения сравнения второй степени.}
\section{Алгоритмы решения сравнений второй степени по примарному и составному модулю.}
\section{Группа, порядок элемента. Теорема Лагранжа.}
.
\section{Нормальный делитель, фактор – группа, первая теорема о гомоморфизме.}
\section{Кольцо многочленов, идеал, теорема Безу, кольцо главных идеалов.}
\section{Конечное поле. Теорема о простом подполе конечного поля. Строение конечного поля. Теорема о примитивном элементе.}
\section{Построение конечных полей. Алгоритм вычисления обратного элемента. Арифметические операции в конечном поле.}
\section{Алгоритмы вычисления дискретного алгоритма.}
\section{Криптосистема Эль - Гамаля. Протокл Диффи - Хеллмана.}
\section{Минимальный многочлен и его свойства. Теорема об изоморфизме конечных полей одной мощности.}
\section{Примитивный многочлен и его свойства. Теорема о разложении многочлена $f(x) = x p^n – x$ на неприводимые многочлены. Критерий принадлежности элемента поля собственному подполю.}
\section{Теорема о группе автоморфизмов конечного поля.}
\section{Рекуррентные последовательности над конечным полем, линейные рекуррентные последовательности (ЛРП). Характеристический и минимальный многочлен ЛРП и их свойства.}
\section{Теорема об определении структуры ЛРП по её характеристическому многочлену. Теорема о ЛРП максимального периода.}
\section{Прямое произведение групп. Теорема о представлении группы в виде прямого произведения своих подгрупп.}
\section{Теорема о примарной абелевой группе.}
.
\section{Теорема о разложении конечной абелевой группы в произведение своих циклических подгрупп.}
\section{Нормализатор, централизатор, класс сопряженных элементов конечной группы. Теорема о числе множеств сопряженных с данным. Теорема о центре примарной группы. Теорема Коши.}
\section{Двойные смежные классы и их свойства. Теорема Силова (первая)}
\section{Вторая и третья теоремы Силова.}
\section{Группы подстановок. Инвариантное множество, орбита. Теорема об индексе стабилизатора группы. Теорема о транзитвности нормализатора подгруппы транзитвной группы. (Ут . 13.4).}
\section{Лемма Бернсайда.}
\section{Регулярные и полурегулярные группы. Порядок полурегулярной группы.}
\section{Блоки и импримитивные группы. Критерий импримитивности. Теорема о импримитивности транзитивной группы с интранзитивным нормальным делителем.}
\section{Примитивные группы. Кратная транзитивность. Критерий кратной транзитивности.}
\section{Теорема о группе автоморфизмов конечной группы.}
\section{Утверждение об изоморфизме стабилизатора и специальной группы автоморфизмов регулярной подгруппы (Ут . 13.5). Утверждение о порядке регулярного нормального делителя кратно транзитивной группы.}
\section{Простая группа. Теорема о простоте знакопеременной группы. Теорема о нормальном делителе симметрической группы.}

\end{document}