\newpage
\chapter{Билеты}
\section{Делимость в кольце целых чисел. НОД, алгоритм Евклида. Критерий взаимной простоты двух чисел.}

\theorem{(о делении с остатком) Пусть $a>0$ и $b>0$ -- целые числа. Тогда $a$ единственным образом представимо в виде 

$$ a = bq + r, \ 0 \le r < b. $$

\noindent Число $q$ -- \textbf{неполное частное}

}

\section{Сравнения и их свойства. Китайская теорема об остатках. Кольцо вычетов. Функция Эйлера и её свойства.}
\section{Теоремы Эйлера и Ферма. Критерий обратимости, алгоритм вычисления обратного элемента.}
\section{Криптографическая теорема (обоснование криптосистемы РСА).}
\section{Теорема о цикличности мультипликативной группы по примарному модулю.}
\section{Решение сравнений первой степени.}
\section{Сравнения второй степени. Символ Лежандра и его свойства.}
\section{Алгоритмы решения сравнений второй степени по простому модулю.}
\section{Символ Якоби и его свойства. Числа Блюма и их свойства. Эквивалентность задачи факторизации и решения сравнения второй степени.}
\section{Алгоритмы решения сравнений второй степени по примарному и составному модулю.}
\section{Группа, порядок элемента. Теорема Лагранжа.}
.
\section{Нормальный делитель, фактор – группа, первая теорема о гомоморфизме.}
\section{Кольцо многочленов, идеал, теорема Безу, кольцо главных идеалов.}
\section{Конечное поле. Теорема о простом подполе конечного поля. Строение конечного поля. Теорема о примитивном элементе.}
\section{Построение конечных полей. Алгоритм вычисления обратного элемента. Арифметические операции в конечном поле.}
\section{Алгоритмы вычисления дискретного алгоритма.}
\section{Криптосистема Эль - Гамаля. Протокл Диффи - Хеллмана.}
\section{Минимальный многочлен и его свойства. Теорема об изоморфизме конечных полей одной мощности.}
\section{Примитивный многочлен и его свойства. Теорема о разложении многочлена $f(x) = x p^n – x$ на неприводимые многочлены. Критерий принадлежности элемента поля собственному подполю.}
\section{Теорема о группе автоморфизмов конечного поля.}
\section{Рекуррентные последовательности над конечным полем, линейные рекуррентные последовательности (ЛРП). Характеристический и минимальный многочлен ЛРП и их свойства.}
\section{Теорема об определении структуры ЛРП по её характеристическому многочлену. Теорема о ЛРП максимального периода.}
\section{Прямое произведение групп. Теорема о представлении группы в виде прямого произведения своих подгрупп.}
\section{Теорема о примарной абелевой группе.}
.
\section{Теорема о разложении конечной абелевой группы в произведение своих циклических подгрупп.}
\section{Нормализатор, централизатор, класс сопряженных элементов конечной группы. Теорема о числе множеств сопряженных с данным. Теорема о центре примарной группы. Теорема Коши.}
\section{Двойные смежные классы и их свойства. Теорема Силова (первая)}
\section{Вторая и третья теоремы Силова.}
\section{Группы подстановок. Инвариантное множество, орбита. Теорема об индексе стабилизатора группы. Теорема о транзитвности нормализатора подгруппы транзитвной группы. (Ут . 13.4).}
\section{Лемма Бернсайда.}
\section{Регулярные и полурегулярные группы. Порядок полурегулярной группы.}
\section{Блоки и импримитивные группы. Критерий импримитивности. Теорема о импримитивности транзитивной группы с интранзитивным нормальным делителем.}
\section{Примитивные группы. Кратная транзитивность. Критерий кратной транзитивности.}
\section{Теорема о группе автоморфизмов конечной группы.}
\section{Утверждение об изоморфизме стабилизатора и специальной группы автоморфизмов регулярной подгруппы (Ут . 13.5). Утверждение о порядке регулярного нормального делителя кратно транзитивной группы.}
\section{Простая группа. Теорема о простоте знакопеременной группы. Теорема о нормальном делителе симметрической группы.}
 
