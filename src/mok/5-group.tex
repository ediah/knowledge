\section{Теория групп}

\task{Определить структуру группы $\mathbb{Z}^*_n$ , разложить её в прямое произведение циклических подгрупп и подсчитать число элементов различного порядка, $n = 84$.}

$|G| = \phi (84) = \phi (2^2) \cdot \phi (3) \cdot \phi (7) = (4 - 2)(3 - 1)(7 - 1) = 24 = 2^3 \cdot 3$.

\noindent Следовательно, $G \cong S_1(8) \times S_2 (3)$.

1.  Определим структуру примарной группы $S_1(8)$. Для этого нужно решить сравнение:

$$x^2 \equiv 1 \pmod{84}.$$

По КТО это равносильно следующей системе:

\begin{equation*}
	\begin{cases}
	x^2 \equiv 1 \pmod 3 \\
	x^2 \equiv 1 \pmod 4 \\
	x^2 \equiv 1 \pmod 7
	\end{cases}
\end{equation*}

$p = 3 = 4 \cdot 0 + 3 \Rightarrow \lege{1}{3} = 1$ -- два решения: $x_1 \equiv \pm 1^{0+1} \pmod 3$.

$p = 4 = 2^2,\ a = 1 \Rightarrow $ два решения: $x_2 = \pm 1 \pmod 4$.

$p = 7 = 4 \cdot 1 + 3 \Rightarrow \lege{1}{7} = 1$ -- два решения: $x_3 \equiv \pm 1^{1+1} \pmod 7$.

Общее решение системы по КТО будет равно:

\begin{multline*}
X = x_1 (4 \cdot 7)[(4 \cdot 7)^{-1} \pmod 3] + x_2 (3 \cdot 7)[(3 \cdot 7)^{-1} \pmod 4] + \\
+ x_3 (3 \cdot 4)[(3 \cdot 4)^{-1} \pmod 7] \pmod {84} =
28 x_1 + 21 x_2 + 36 x_3 \pmod {84}
\end{multline*}

Таким образом, множество элементов второго порядка группы $G$ это:

$$M_2 = \{13, 43, 55, 29, 41, 71, 83\}.$$

Обозначим за $A_i = \left\langle M_{2, i} \right\rangle $, где $M_{2, i}$ -- $i$-тый элемент множества $M_2$. Тогда $S_1(8) \cong A_i \times A_j \times A_k$, $i, j, k = \overline{1, 7} $.

2. Определим структуру примарной группы $S_2(3)$. Для этого нужно решить сравнение:

$$x^3 \equiv 1 \pmod{84}.$$

По КТО это равносильно следующей системе:

\begin{equation*}
	\begin{cases}
	x^3 \equiv 1 \pmod 3 \\
	x^3 \equiv 1 \pmod 4 \\
	x^3 \equiv 1 \pmod 7
	\end{cases}
\end{equation*}

Первые два сравнения, очевидно, имеют единственные решения $x_1 =~1$, $x_2 = 1$. Последнее сравнение имеет 3 решения: $x_3 = \{1, 2, 4\}$. Общее решение исходного сравнения будет равно:

$$X = 28 x_1 + 21 x_2 + 36 x_3 \pmod {84} = 49 + 36 x_3 \pmod {84}$$

Множество элементов третьего порядка группы $G$ это:

$$M_3 = \{25, 37\}$$

Таким образом, $S_2(3) = \left\langle 25 \right\rangle = \left\langle 37 \right\rangle$. \\

\noindent Ответ: $G \cong A_i \times A_j \times A_k \times \left\langle 25 \right\rangle = A_i \times A_j \times A_k \times \left\langle 37 \right\rangle$, $i, j, k = \overline{1, 7}$. 
