\section{Теория групп}

\task{Определить структуру группы $\mathbb{Z}^*_n$ , разложить её в прямое произведение циклических подгрупп и подсчитать число элементов различного порядка, $n = 84$.}

$|G| = \phi (84) = \phi (2^2) \cdot \phi (3) \cdot \phi (7) = (4 - 2)(3 - 1)(7 - 1) = 24 = 2^3 \cdot 3$.

\noindent Следовательно, $G \cong S_1(8) \times S_2 (3)$.

1.  Определим структуру примарной группы $S_1(8)$. Для этого нужно решить сравнение:

$$x^2 \equiv 1 \pmod{84}.$$

По КТО это равносильно следующей системе:

\begin{equation*}
	\begin{cases}
	x^2 \equiv 1 \pmod 3 \\
	x^2 \equiv 1 \pmod 4 \\
	x^2 \equiv 1 \pmod 7
	\end{cases}
\end{equation*}

$p = 3 = 4 \cdot 0 + 3 \Rightarrow \lege{1}{3} = 1$ -- два решения: $x_1 \equiv \pm 1^{0+1} \pmod 3$.

$p = 4 = 2^2,\ a = 1 \Rightarrow $ два решения: $x_2 = \pm 1 \pmod 4$.

$p = 7 = 4 \cdot 1 + 3 \Rightarrow \lege{1}{7} = 1$ -- два решения: $x_3 \equiv \pm 1^{1+1} \pmod 7$.

Общее решение системы по КТО будет равно:

\begin{multline*}
X = x_1 (4 \cdot 7)[(4 \cdot 7)^{-1} \pmod 3] + x_2 (3 \cdot 7)[(3 \cdot 7)^{-1} \pmod 4] + \\
+ x_3 (3 \cdot 4)[(3 \cdot 4)^{-1} \pmod 7] \pmod {84} =
28 x_1 + 21 x_2 + 36 x_3 \pmod {84}
\end{multline*}

Таким образом, множество элементов второго порядка группы $G$ это:

$$M_2 = \{13, 43, 55, 29, 41, 71, 83\}.$$

Обозначим за $A_i = \left\langle M_{2, i} \right\rangle $, где $M_{2, i}$ -- $i$-тый элемент множества $M_2$. Тогда $S_1(8) \cong A_i \times A_j \times A_k$, $i, j, k = \overline{1, 7} $.

2. Определим структуру примарной группы $S_2(3)$. Для этого нужно решить сравнение:

$$x^3 \equiv 1 \pmod{84}.$$

По КТО это равносильно следующей системе:

\begin{equation*}
	\begin{cases}
	x^3 \equiv 1 \pmod 3 \\
	x^3 \equiv 1 \pmod 4 \\
	x^3 \equiv 1 \pmod 7
	\end{cases}
\end{equation*}

Первые два сравнения, очевидно, имеют единственные решения $x_1 =~1$, $x_2 = 1$. Последнее сравнение имеет 3 решения: $x_3 = \{1, 2, 4\}$. Общее решение исходного сравнения будет равно:

$$X = 28 x_1 + 21 x_2 + 36 x_3 \pmod {84} = 49 + 36 x_3 \pmod {84}$$

Множество элементов третьего порядка группы $G$ это:

$$M_3 = \{25, 37\}$$

Таким образом, $S_2(3) = \left\langle 25 \right\rangle = \left\langle 37 \right\rangle$. \\

\noindent Ответ: $G \cong A_i \times A_j \times A_k \times \left\langle 25 \right\rangle = A_i \times A_j \times A_k \times \left\langle 37 \right\rangle$, $i, j, k = \overline{1, 7}$. 

\task{Доказать, что $y = x^3$ -- подстановка над $\mathbb{Z}_p$, если \\$p \equiv 2 \pmod{3}$, то есть, 
$$x_1^3 \equiv x_2^3 \pmod p \Rightarrow x_1 \equiv x_2 \pmod p.$$}

Поле $\mathbb{Z}_p$ имеет порядок $\phi(p) = p - 1$.
Из $x_1^3 \equiv x_2^3 \pmod p$ получим, что $(x_1 x_2 ^{-1})^3 \equiv 1 \pmod p$. Тогда порядок $x_1 x_2^{-1}$ является делителем $3$. Существует два варианта:

1. Ord($x_1 x_2^{-1}$) = 1. Это означает, что $x_1 \equiv x_2 \pmod p$.

2. Ord($x_1 x_2^{-1}$) = 3. Так как $p = 3m + 2$, то $\phi(p) = 3m + 1$. То есть, порядок элемента не является делителем порядка группы, что противоречит теореме Лагранжа.

Следовательно, $x_1 \equiv x_2 \pmod p$. А значит, $y = x^3$ -- подстановка над $\mathbb{Z}_p$.

\task{Найти порядок и цикловое представление подстановки $y~=~x^3,\ p = 11$.}

\medskip

\begin{tabular}{||c||c|c|c|c|c|c|c|c|c|c|c|}
 $x$ & 0 & 1 & 2 & 3 & 4 & 5 & 6 & 7 & 8 & 9 & 10 \\
 \hline
 $y$ & 0 & 1 & 8 & 5 & 9 & 4 & 7 & 2 & 6 & 3 & 10
\end{tabular}

\medskip

Получим 5 циклов: (0), (1), (2, 8, 6, 7), (3, 5, 4, 9), (10). Порядок подстановки равен НОК длины циклов: НОК$(1, 1, 4, 4, 1) = 4$.

Ответ: Порядок подстановки (0)(1)(2, 8, 6, 7)(3, 5, 4, 9)(10) равен 4.

\task{Найти порядок подстановки $y = 5 x + 3 \pmod {12}$}

\medskip

\begin{tabular}{||c||c|c|c|c|c|c|c|c|c|c|c|c|}
 $x$ & 0 & 1 & 2 & 3 & 4 & 5 & 6 & 7 & 8 & 9 & 10 & 11 \\
 \hline
 $y$ & 3 & 8 & 1 & 6 & 11 & 4 & 9 & 2 & 7 & 0 & 5 & 10
\end{tabular}

\medskip

Получим 3 цикла: (0, 3, 6, 9), (1, 8, 7, 2), (4, 11, 10, 5). Порядок подстановки равен НОК длины циклов: НОК$(4, 4, 4) = 4$.

Ответ: Порядок подстановки равен 4.