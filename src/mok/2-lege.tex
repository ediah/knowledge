\section{Квадратичные вычеты, сравнения, символ Лежандра.}

Докажем вспомогательные леммы.

\lemmata{Если $p = 2^m + 1$ -- простое и $\lege{a}{p} = -1$, то $\gen{a} = \mathbb{Z}^*_p$.}
\prove{
По определению первообразного корня достаточно доказать два утверждения: $\ a^{\phi(p)} = a^{2^m} \equiv 1 \pmod p$ и $\ a^{\frac{\phi(p)}{2}} = a^{2^{m-1}} \not \equiv 1 \pmod p.$

$$ a ^ {2 ^ {m - 1}} = a ^ {\frac{p - 1}{2}} = \lege{a}{p} = -1 \not \equiv 1 \pmod p, $$
$$a ^ {2^m} = (a ^ {2 ^ {m - 1}} )^2 = (-1)^2 = 1 \equiv 1 \pmod p. $$
}

\lemmata{Если число $p = 2^m + 1$ -- простое, $m > 1$, то $p \equiv 2 \pmod 3.$}
\prove{
По теореме о делении с остатком, число $p$ представимо в виде:
$$p = 3k + t, 0 \le t < 3.$$

\noindent Рассмотрим данное равенство при различных $t$.

а) $t = 0 \Rightarrow p = 3k$, то есть, $p$ не является простым числом при $k > 1$ (а значит, при $m > 1$). Противоречие $\Rightarrow t \ne 0$.

б) $t = 1 \Rightarrow 2^m = 3k$ -- этого не может быть ни при каком целом $k$ по лемме Евклида (по крайней мере один из сомножителей числа $2^m$ должен делиться на $3$). Следовательно, $t \ne 1$.

Тогда $t = 2$ -- единственный вариант, $p = 3k + 2$.
}

\lemmata{Если $p = 2^{2^n} + 1$, $n > 1$, то $p \equiv 2 \pmod 5.$}
\prove{
Докажем по индукции.

1) При $n = 2$ утверждение верно: $2^{2^2}+1 = 17 \equiv 2 \pmod 5$.

2) Пусть для $n = m$ верно, докажем для $n = m + 1$:
$$2^{2^{m+1}} + 1 = (2^{2^{m}} + 1 - 1)^2 + 1 = (2 - 1)^2 + 1 = 2 \equiv 2 \pmod 5.$$
}

\lemmata{Если $p = 2^{2^n} + 1$, $n = 2k$, то $p \equiv 3 \pmod 7.$}
\prove{
Докажем по индукции.

1) При $k = 0$ утверждение верно: $2^{2^0}+1 = 3 \equiv 3 \pmod 7$.

2) Пусть для $k = m$ верно, докажем для $k = m + 1$:
$$2^{2^{2(m+1)}} + 1 = (2^{2^{2m}} + 1 - 1)^4 + 1 = (3 - 1)^4 + 1 = 17 \equiv 3 \pmod 7$$
}

\lemmata{Если $p = 2^{2^n} + 1$, $n = 2k + 1$, то $p \equiv 5 \pmod 7.$}
\prove{
Докажем по индукции.

1) При $k = 0$ утверждение верно: $2^{2^1}+1 = 5 \equiv 5 \pmod 7$.

2) Пусть для $k = m$ верно, докажем для $k = m + 1$:
$$2^{2^{2(m+1)+1}} + 1 = (2^{2^{2m+1}} + 1 - 1)^4 + 1 = (5 - 1)^4 + 1 = 257 \equiv 5 \pmod 7$$
}

\task{Доказать, что сравнение $x^2 + 1 \equiv 0 \pmod p$ разрешимо тогда и только тогда, когда $p \equiv 1 \pmod 4$.}

$x^2 + 1 \equiv 0 \pmod p $ -- разрешимо $ \Leftrightarrow \lege{-1}{p} = 1 \Leftrightarrow (-1)^{\frac{p-1}{2}} = 1 \\ \Leftrightarrow \frac{p-1}{2} = 2k \Leftrightarrow p = 4k + 1 \Leftrightarrow p \equiv 1 \pmod 4$

\task{Доказать, что сравнение $x^2 + 2 \equiv 0 \pmod p$ разрешимо тогда и только тогда, когда $p = 1, 3 \pmod 8$.}

$x^2 + 2 \equiv 0 \pmod p $ -- разрешимо $ \Leftrightarrow \lege{-2}{p} = 1. \Leftrightarrow \big \{ \lege{-2}{p} = \lege{-1}{p} \cdot \\ \cdot \lege{2}{p} = (-1)^{\frac{p-1}{2}} \cdot (-1)^{\frac{p^2-1}{8}} \big \} \Leftrightarrow \frac{p-1}{2} + \frac{p^2-1}{8} = 2k \Leftrightarrow p^2 + 4p - 16k - 5 = 0.$

Представим $p$, используя теорему о делении с остатком, в следующем виде: $p = 8m + t,\ 0 \le t < 8$. Решим полученную систему относительно $t$.

$(8m + t)^2 + 4(8m + t) - 16k -5 = 0$

$ t^2 + (16k + 4)t + 64k^2 + 32k - 16m - 5 = 0$

$t_{1, 2} = -8k - 2 \pm \sqrt{16m + 9} \pmod 8 = -2 \pm 3 \pmod 8 \Rightarrow t = 1,3$ \\

\noindent Тогда $p^2 + 4p - 16k - 5 = 0 \Leftrightarrow p = 1, 3 \pmod 8$.

\task{Доказать, что сравнение $x^2 + 3 \equiv 0 \pmod p$ разрешимо тогда и только тогда, когда $p \equiv 1 \pmod 6$.}

Пусть $p = 3k + t, t < 3.$

$x^2 + 3 \equiv 0 \pmod p \Leftrightarrow \lege{-3}{p} = 1.$

$ \lege{-3}{p} = \lege{-1}{p} \lege{3}{p} = (-1)^{\frac{p-1}{2}}(-1)^{\frac{p-1}{2} \cdot \frac{3-1}{2}} \lege{p}{3} = (-1)^{3k + t -1} \lege{t}{3}$\\

а) $t = 0 \Rightarrow \lege{0}{3} = 0$, $(-1)^{3k + t -1} \lege{t}{3} = 0 \ne 1$

б) $t = 1 \Rightarrow \lege{1}{3} = 1$, $(-1)^{3k + t -1} \lege{t}{3} = (-1)^{3k} \cdot 1 = (-1)^{3k}$.

в) $t = 2 \Rightarrow \lege{2}{3} = -1$, $(-1)^{3k + t -1} \lege{t}{3} = (-1)^{3k + 1} \cdot (-1) = (-1)^{3k}$ \\

$(-1)^{3k} = 1 \Leftrightarrow k = 2m \Leftrightarrow p = 6m + 1 \Leftrightarrow p \equiv 1 \pmod 6$

\task{Доказать, что если $p = 2^n + 1$ -- простое, $n > 2$, то $\lege{3}{p} = -1$ и $\gen{3} = \mathbb{Z}^*_p$.}

$p = 3k + 2$ по лемме 2.2. 

$ \lege{3}{p} = (-1)^{\frac{3-1}{2} \cdot \frac{2^n + 1 -1}{2}} \lege{p}{3} = (-1)^{2^{n - 1}} \lege{2}{3} = -1$ \\

Выполнены все условия леммы 2.1 $\Rightarrow \gen{3} = \mathbb{Z}^*_p$.

\task{Доказать, что если $p = 2^n + 1$ -- простое и $\lege{a}{p} = -1$, то $\gen{a} = \mathbb{Z}^*_p$.}

Доказано в качестве леммы 2.1.

\task{Доказать, что если $p = 4q + 1$, $p$ и $q$ -- простые, то $\gen{2} = \mathbb{Z}^*_p$.}

По определению первообразного корня достаточно доказать три утверждения: \\ 1) $2^{\phi(p)} = 2^{4q} \equiv 1 \pmod p$,\\ 2)$\ 2^{\frac{\phi(p)}{2}} =  2^{2q} \not \equiv 1 \pmod p$, \\ 3) $2^{\frac{\phi(p)}{q}} =  2^{4} \not \equiv 1 \pmod p$.

Начнём с третьего. Представим $2^4$ в следующем виде: $2^4 = pk + t, \\ 0 \le t < p$. Значит, нам нужно доказать, что $t \ne 1$. Предположим, что это не так, тогда $pk = 2^4 - 1 = 15$. Обратим внимание на условие: если и $p$, и $q$ -- простые числа, то $p$ не может быть ни 3, ни 5. Значит, в левой части равенства содержится простой множитель, которого нет в правой части. Мы получили противоречие, а значит, $t \ne 1 \Rightarrow 2^{\frac{\phi(p)}{q}} =  2^{4} \not \equiv 1 \pmod p$.

Рассмотрим теперь второе утверждение. Заметим, что:

$$\lege{2}{4q + 1} = 2 ^ {\frac{4q + 1 - 1}{2}} = 2 ^ {2q} \pmod {4q + 1}.$$

Вычислим $\lege{2}{4q+1} = (-1) ^ {\frac{(4q+1)^2 - 1}{8}} = (-1)^{2q^2 + q} = \big\{q$ -- нечет$\big\} = -1$. Тем самым мы доказали второе утверждение.

Поскольку $2^{4q} = (2^{2q})^2 = (-1)^2 = 1 \pmod {4q+1}$, то первое утверждение становится следствием второго.

\task{Доказать, что если $p = 2^{2^n} + 1$ -- простое и $\lege{a}{p} = -1$, то $\gen{a} = \mathbb{Z}^*_p$.}

Приняв $m = 2^n$ в лемме 2.1, получим справедливость данного утверждения.


\task{Доказать, что если $p = 2^{2^n} + 1$ -- простое, $n>2$, то $\gen{3} = \gen{5} = \gen{7} = \mathbb{Z}^*_p$.}

Покажем $\lege{3}{p} = \lege{5}{p} = \lege{7}{p} = -1$.

$2^{2^n} + 1 = 3k + 2$ по лемме 2.2.
$$\lege{3}{p} = (-1) ^ {\frac{3 - 1}{2} \cdot \frac{2^{2^n} + 1 - 1}{2}} \lege{p}{3} = (-1) ^ {2^{2^n - 1}} \lege{3k+2}{3} = \lege{2}{3} = 2^{\frac{3-1}{2}} \pmod 3 = -1$$

$2^{2^n} + 1 = 5k + 2$ по лемме 2.3.
$$\lege{5}{p} = (-1) ^ {\frac{5 - 1}{2} \cdot \frac{2^{2^n} + 1 - 1}{2}} \lege{p}{5} = (-1) ^ {2^{2^n}} \lege{5k+2}{5} = \lege{2}{5} = 2^{\frac{5-1}{2}} \pmod 5= -1$$

$2^{2^n} + 1 = 7k + 3,\ n = 2t$ по лемме 2.4.

$$\lege{7}{p} = (-1) ^ {\frac{7 - 1}{2} \cdot \frac{2^{2^n} + 1 - 1}{2}} \lege{p}{7} = (-1) ^ {2^{2^n}} \lege{7k+3}{7} = \lege{3}{7} = 3^{\frac{7-1}{2}} \pmod 7 = -1$$

$2^{2^n} + 1 = 7k + 5,\ n = 2t + 1$ по лемме 2.5.

$$\lege{7}{p} = \lege{5}{7} = 5^{\frac{7-1}{2}} \pmod 7 = -1.$$

Осталось применить лемму 2.1, и исходное утверждение будет доказано. 
