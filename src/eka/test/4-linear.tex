\section{Линейный криптоанализ}

Поскольку мы не связаны никакими ограничениями в выборе тех или иных функций, поисследуем, как изменится эффективность метода Мицуру Мацуи, если допустить хотя бы малейшие необдуманные изменения в оригинальном алгоритме DES.

Пусть функция расширения будет иметь вид: 
$$E(\vec{X}) = (X[4], X[3], X[1], X[3], X[2], X[4], X[6], X[7], X[5], X[7], X[8], X[6])$$

Функция перестановки:
$$P(\vec{X}) = (X[2], X[6], X[4], X[7], X[3], X[8], X[5], X[1])$$

Возьмём следующие S-боксы:

\medskip

\begin{tabular}{c||c|c|c|c|c|c|c|c|c|c|c|c|c|c|c|c}
& 0 & 1 & 2 & 3 & 4 & 5 & 6 & 7 & 8 & 9 & 10 & 11 & 12 & 13 & 14 & 15 \\
\hline
\hline
0 & 13 & 4 & 14 & 1 & 2 & 11 & 15 & 8 & 3 & 10 & 6 & 12 & 7 & 9 & 0 & 5 \\
\hline
1 & 0 & 8 & 7 & 4 & 14 & 2 & 13 & 9 & 10 & 6 & 12 & 11 & 1 & 5 & 3 & 15 \\
\hline
2 & 4 & 0 & 14 & 8 & 13 & 6 & 2 & 11 & 15 & 12 & 9 & 7 & 3 & 10 & 5 & 1 \\
\hline
3 & 13 & 12 & 8 & 2 & 4 & 9 & 1 & 7 & 5 & 3 & 11 & 14 & 10 & 0 & 6 & 15 \\
\end{tabular}

\medskip

\begin{tabular}{c||c|c|c|c|c|c|c|c|c|c|c|c|c|c|c|c}
    & 0 & 1 & 2 & 3 & 4 & 5 & 6 & 7 & 8 & 9 & 10 & 11 & 12 & 13 & 14 & 15 \\
    \hline
    \hline
    0 & 10 & 1 & 8 & 14 & 6 & 11 & 3 & 4 & 9 & 12 & 2 & 13 & 7 & 0 & 5 & 15 \\
    \hline
    1 & 3 & 5 & 4 & 7 & 15 & 2 & 8 & 14 & 12 & 0 & 1 & 10 & 6 & 9 & 11 & 13 \\
    \hline
    2 & 0 & 2 & 7 & 11 & 10 & 4 & 13 & 1 & 5 & 8 & 12 & 6 & 9 & 3 & 14 & 15 \\
    \hline
    3 & 13 & 11 & 10 & 1 & 3 & 15 & 4 & 2 & 8 & 6 & 7 & 12 & 5 & 0 & 14 & 9 \\
\end{tabular}

Построим таблицу значений $NS_1(\alpha, \beta):$

\begin{tabular}{c||c|c|c|c|c|c|c|c|c|c|c|c|c|c|c|c}
& 1 & 2 & 3 & 4 & 5 & 6 & 7 & 8 & 9 & 10 & 11 & 12 & 13 & 14 & 15 \\
\hline
\hline
1 & 0 & 0 & 0 & 0 & 0 & 0 & 0 & 0 & 0 & 0 & 0 & 0 & 0 & 0 & 0 \\
\hline
2 & -2 & -2 & -4 & -4 & 2 & -2 & 4 & 2 & 0 & 0 & 6 & 6 & -4 & 0 & -2 \\
\hline
3 & -2 & -6 & 0 & -4 & 2 & 2 & 0 & 2 & 8 & -4 & 2 & 6 & 4 & -12 & 2 \\
\hline
4 & 4 & 2 & -2 & 2 & 2 & -8 & 0 & 2 & 2 & 0 & 8 & 0 & -4 & 2 & 6 \\
\hline
5 & -4 & -6 & 6 & -2 & -2 & -4 & -4 & -2 & 6 & -4 & 4 & 0 & -4 & 2 & -2 \\
\hline
6 & -6 & 0 & 2 & -2 & 4 & -2 & -4 & 0 & -2 & 4 & 2 & 2 & -4 & -2 & -8 \\
\hline
7 & -6 & 4 & 6 & 2 & 0 & -2 & 4 & 4 & -6 & -12 & -6 & 2 & -4 & 2 & -4 \\
\hline
8 & 8 & 2 & 6 & -4 & -4 & -2 & 2 & -4 & -4 & -2 & 2 & 0 & 0 & 2 & -2 \\
\hline
9 & 0 & 2 & -2 & -4 & -4 & 6 & -6 & 0 & 0 & 2 & 6 & 4 & -4 & -2 & 2 \\
\hline
10 & -6 & 0 & -2 & 0 & 2 & 4 & 2 & -6 & 4 & 2 & 0 & -6 & 4 & -2 & 4 \\
\hline
11 & 2 & -4 & -6 & 0 & -14 & 0 & 6 & -2 & 0 & 2 & 0 & -2 & -8 & -2 & -4 \\
\hline
12 & -4 & 4 & 4 & -2 & 6 & -2 & 2 & 2 & 2 & 2 & -2 & 4 & -8 & 0 & 8 \\
\hline
13 & 4 & 4 & 4 & 2 & 2 & -6 & -2 & 2 & 2 & 10 & -2 & 0 & 4 & -4 & -4 \\
\hline
14 & -2 & 2 & -4 & 2 & 4 & -4 & 2 & 4 & 6 & 2 & 0 & 2 & 0 & 0 & 2 \\
\hline
15 & -2 & -2 & 8 & -2 & 0 & 4 & -6 & -4 & -2 & -2 & -4 & -2 & -4 & 0 & 2 \\
\hline
16 & 6 & 6 & 0 & 0 & 2 & 2 & 0 & -2 & 8 & -4 & 2 & 2 & 0 & -4 & -18 \\
\hline
17 & 2 & -6 & 0 & 0 & -2 & -2 & 8 & -2 & 4 & 8 & -6 & 2 & -4 & 0 & -2 \\
\hline
18 & 0 & 0 & -4 & -8 & 4 & 0 & 0 & -4 & 0 & 4 & 4 & 8 & 0 & 0 & -4 \\
\hline
19 & 4 & 0 & -8 & 0 & 0 & -8 & -12 & -4 & -4 & -4 & 0 & 0 & 4 & 0 & 0 \\
\hline
20 & 2 & 4 & 2 & -6 & -4 & -2 & 4 & 0 & 10 & 0 & -2 & 2 & 4 & 2 & 0 \\
\hline
21 & -2 & 8 & -6 & -2 & 4 & -2 & 0 & 12 & 2 & 0 & 2 & -6 & 0 & 6 & 0 \\
\hline
22 & 4 & 6 & -2 & 2 & -2 & -4 & 4 & 2 & 6 & -4 & 4 & -4 & 0 & 2 & 2 \\
\hline
23 & 0 & 6 & -6 & 6 & 6 & 0 & -4 & -10 & -2 & 0 & -4 & -4 & -4 & 2 & -2 \\
\hline
24 & 2 & 8 & 2 & 0 & 6 & 4 & 2 & 6 & -4 & 6 & 4 & 2 & -4 & -2 & 0 \\
\hline
25 & -2 & 4 & -6 & 0 & -6 & 0 & 2 & 2 & -4 & 6 & 8 & -2 & 4 & -2 & -4 \\
\hline
26 & 0 & -6 & 2 & 0 & -4 & 2 & 6 & 0 & -4 & -6 & 6 & -4 & 4 & 6 & -2 \\
\hline
27 & 4 & 2 & 6 & -8 & 0 & -6 & 2 & -4 & -4 & 6 & -2 & 0 & 4 & 2 & -2 \\
\hline
28 & -2 & 6 & 4 & 2 & 0 & 0 & -2 & -4 & 2 & -2 & 4 & -2 & 4 & 8 & -2 \\
\hline
29 & 2 & -6 & 4 & -2 & 0 & 0 & -6 & 4 & 6 & -6 & 4 & -6 & -4 & -8 & 2 \\
\hline
30 & -4 & 0 & -4 & 2 & -2 & -2 & -6 & 2 & 6 & -2 & -6 & 4 & -8 & 4 & 0 \\
\hline
31 & 0 & 0 & 0 & -2 & -2 & 2 & 2 & 2 & -6 & -2 & -2 & -8 & 0 & 0 & 0 \\
\hline
32 & 0 & 0 & 0 & 0 & 0 & 0 & 0 & 0 & 0 & 0 & 0 & 0 & 0 & 0 & 0 \\
\hline
33 & 0 & 0 & 0 & 0 & 0 & 0 & 0 & 0 & 0 & 0 & 0 & 0 & 0 & 0 & 0 \\
\hline
34 & 2 & -6 & 4 & 0 & 2 & 6 & 0 & 2 & -4 & -4 & 6 & 10 & 4 & 8 & 2 \\
\hline
35 & 2 & -2 & 0 & 0 & 2 & 2 & 4 & 2 & 4 & 0 & -6 & -6 & -4 & 4 & -2 \\
\hline
36 & 0 & -6 & -6 & 6 & 2 & 4 & 0 & 2 & -2 & 0 & -4 & -4 & 4 & -2 & -10 \\
\hline
37 & -8 & 2 & 2 & 2 & -2 & -8 & -4 & -2 & 2 & -4 & 8 & -4 & 4 & -2 & -2 \\
\hline
38 & 2 & 4 & -2 & 6 & -4 & 2 & 0 & 0 & -2 & 0 & -2 & 2 & 4 & 2 & 4 \\
\hline
39 & 2 & 0 & -6 & -6 & 8 & -6 & 0 & 4 & -6 & -8 & -2 & 2 & 4 & -2 & 0 \\
\hline
40 & 4 & -2 & -2 & 0 & -4 & -2 & -2 & 4 & 0 & 2 & 2 & -4 & -8 & 10 & 2
\end{tabular}

\begin{tabular}{c|c|c|c|c|c|c|c|c|c|c|c|c|c|c|c|c}
    & 1 & 2 & 3 & 4 & 5 & 6 & 7 & 8 & 9 & 10 & 11 & 12 & 13 & 14 & 15 \\
    \hline
41 & -4 & -2 & -10 & 0 & -4 & -10 & 6 & 0 & -4 & -2 & -2 & 8 & -4 & -2 & -2 \\
\hline
42 & 2 & 0 & -2 & 0 & 2 & -4 & 2 & 2 & -4 & -6 & 0 & 2 & 4 & -2 & 4 \\
\hline
43 & -6 & 4 & 2 & 0 & 2 & 0 & 14 & -2 & 0 & -6 & 0 & -2 & 0 & -2 & -4 \\
\hline
44 & 4 & 0 & 0 & 6 & 6 & 2 & 6 & -6 & 2 & -2 & 10 & -4 & -8 & -4 & 4 \\
\hline
45 & -4 & 0 & 0 & -6 & 2 & -2 & 2 & 2 & -6 & -2 & 2 & 0 & -4 & 0 & 0 \\
\hline
46 & -6 & 2 & 8 & 6 & -4 & -8 & 2 & -4 & 2 & 2 & 4 & 6 & 0 & 4 & 2 \\
\hline
47 & -6 & 6 & -4 & 2 & -8 & 8 & 2 & -4 & 2 & -2 & 0 & 10 & 4 & 4 & 2 \\
\hline
    48 & -2 & -2 & 0 & 0 & 2 & 2 & 0 & -2 & 0 & 4 & 2 & 2 & 0 & -4 & -2 \\
    \hline
    49 & -6 & 2 & 0 & 0 & -2 & -2 & -8 & -2 & -4 & 0 & -6 & 2 & -4 & 0 & -2 \\
    \hline
    50 & -4 & -4 & -4 & -4 & 4 & 0 & 4 & -4 & 4 & 0 & -4 & -4 & 8 & 0 & 8 \\
    \hline
    51 & 0 & 4 & 0 & 4 & 0 & 0 & 0 & -4 & 0 & 0 & 0 & 4 & -4 & -8 & 4 \\
\hline
    52 & 6 & -4 & 6 & 6 & 4 & -6 & 4 & 0 & -2 & 0 & -6 & 6 & 4 & -2 & 0 \\
    \hline
    53 & 2 & 0 & -2 & -6 & -4 & -6 & 0 & -4 & 6 & 0 & -2 & -2 & 0 & 2 & 0 \\
    \hline
    54 & 4 & 2 & -6 & 2 & -2 & 8 & 0 & 2 & -2 & 0 & 0 & 4 & 0 & -2 & 6 \\
    \hline
    55 & 0 & -6 & -2 & 6 & 6 & 4 & 0 & 6 & 6 & -4 & 0 & 4 & -4 & 6 & -6 \\
    \hline
    56 & -10 & 4 & 2 & -4 & -2 & 4 & -2 & 6 & 0 & 2 & 4 & -2 & 4 & -2 & -4 \\
    \hline
    57 & 2 & 0 & -6 & -4 & 2 & 0 & -2 & -6 & 8 & -6 & 0 & 2 & 4 & 6 & 0 \\
    \hline
    58 & 0 & 2 & 2 & 8 & 4 & 2 & -2 & 0 & 4 & 2 & -2 & 4 & 4 & 6 & -2 \\
    \hline
    59 & 4 & 2 & -2 & 0 & 8 & 2 & 2 & -12 & -4 & -2 & 6 & 0 & -4 & 2 & -2 \\
    \hline
    60 & -2 & 10 & 0 & -6 & 0 & 4 & 2 & -4 & 2 & -6 & -8 & -2 & -4 & -4 & 2 \\
    \hline
    61 & 2 & -2 & 0 & -10 & 0 & 4 & -2 & -4 & -2 & -2 & 0 & 2 & -4 & 4 & -2 \\
    \hline
    62 & 0 & 0 & 0 & -10 & 6 & 2 & 2 & 2 & 2 & 6 & -2 & 0 & 0 & 8 & 0 \\
    \hline
    63 & -12 & -8 & -4 & 2 & 6 & -2 & 2 & -6 & -2 & 6 & 2 & -4 & 0 & 4 & 0
\end{tabular}

\medskip

Наибольшее по модулю число в этой таблице находится на позиции (16, 15), оно равно -18. Тогда уравнение
$$\vec{X}[2] \oplus \vec{Y}[1, 2, 3, 4] = \vec{K}[2]$$
является эффективным линейным статистическим аналогом 1-го S-бокса в классе всех линейных статистических аналогов вида
$$(\vec{Y}, \vec{j}) = (\vec{X} \oplus \vec{K}, \vec{i})$$
с вероятностью $p_1 = \frac{-18 + 32}{64} = \frac{7}{32}$ и $\Delta_1= |1 - \frac{7}{16}| = \frac{9}{16}$

Повторим то же самое со вторым S-боксом:

\begin{tabular}{c||c|c|c|c|c|c|c|c|c|c|c|c|c|c|c|c}
    & 1 & 2 & 3 & 4 & 5 & 6 & 7 & 8 & 9 & 10 & 11 & 12 & 13 & 14 & 15 \\
    \hline
    \hline
    1 & 0 & 0 & 0 & 0 & 0 & 0 & 0 & 0 & 0 & 0 & 0 & 0 & 0 & 0 & 0 \\
    \hline
2 & 0 & 0 & 0 & -2 & 2 & -2 & 2 & 0 & -8 & -4 & 4 & -6 & -2 & -2 & 2 \\
\hline
3 & 0 & 0 & -8 & 2 & 6 & 2 & -2 & 0 & 0 & -4 & 4 & -2 & 10 & 2 & 6 \\
\hline
4 & 0 & 2 & 2 & 6 & 2 & 0 & 4 & 4 & 4 & -2 & -2 & -2 & 2 & 0 & 12 \\
\hline
5 & 4 & 2 & -2 & 6 & -2 & 0 & -8 & 0 & -4 & 2 & 6 & -6 & 2 & 4 & -4 \\
\hline
6 & -4 & -2 & -6 & -4 & 0 & 2 & -2 & -4 & 0 & -2 & 2 & 0 & -4 & 2 & 6 \\
\hline
7 & 0 & -2 & -2 & 0 & 0 & 6 & -2 & 0 & 8 & 10 & 2 & -8 & 0 & 2 & 2 \\
\hline
8 & 4 & 4 & 0 & 2 & 2 & -2 & -2 & 0 & -8 & -4 & -4 & -2 & 2 & 10 & -2 \\
\hline
9 & 4 & 0 & -4 & 2 & 2 & 2 & 2 & -4 & 4 & -4 & -4 & 2 & 6 & -6 & -2 \\
\hline
10 & 0 & 4 & 4 & 4 & 4 & 0 & -8 & 0 & 4 & 0 & 4 & 4 & 0 & -4 & 0 \\
\hline
11 & 0 & 0 & 8 & 0 & 0 & 0 & 0 & 4 & 0 & 8 & -4 & -4 & 0 & 0 & 4 \\
\hline
12 & 0 & 6 & -2 & 0 & 0 & -2 & 6 & -4 & 0 & 2 & -2 & 4 & 0 & -6 & -2 \\
\hline
13 & -4 & 2 & -2 & 0 & 4 & 2 & -10 & 4 & -4 & 6 & -2 & 4 & -4 & -2 & 6 \\
\hline
14 & 0 & 2 & 2 & 2 & -2 & -4 & 8 & -4 & 0 & 2 & 6 & 2 & 2 & 0 & 0 \\
\hline
15 & -4 & -2 & -6 & -2 & -2 & -4 & 12 & 4 & 4 & 6 & 6 & -2 & 2 & 0 & 4 \\
\hline
16 & 0 & -6 & 2 & 4 & 0 & -6 & -2 & 4 & 0 & 6 & 2 & 0 & -8 & -2 & -10 \\
\hline
17 & 4 & 2 & -2 & 4 & -4 & 2 & 2 & 0 & 0 & 2 & 2 & -4 & 0 & 10 & -2 \\
\hline
18 & 4 & 2 & 6 & 2 & 6 & 0 & 4 & -4 & 4 & -6 & 10 & 2 & -14 & 4 & 12 \\
\hline
19 & 0 & -6 & 2 & -2 & 6 & 4 & 4 & 0 & -4 & -2 & -6 & 2 & -2 & 4 & 0 \\
\hline
20 & -4 & -8 & -4 & -2 & 2 & -6 & -2 & -4 & 4 & 4 & -4 & 6 & -2 & 2 & 2 \\
\hline
21 & 12 & 0 & -4 & -2 & 2 & 2 & -2 & 4 & 4 & 4 & -4 & -2 & -2 & 2 & 2 \\
\hline
22 & 4 & -4 & 0 & 4 & -4 & 4 & 4 & 4 & -4 & 4 & 4 & 8 & 4 & 4 & 0 \\
\hline
23 & 12 & 4 & 0 & 0 & 0 & -8 & 0 & -4 & -4 & 4 & -4 & -4 & 0 & 0 & 4 \\
\hline
24 & -8 & -2 & -2 & 2 & 2 & 4 & 4 & 0 & 0 & 6 & 6 & -2 & 6 & 0 & 0 \\
\hline
25 & 4 & -6 & 6 & 2 & 6 & 8 & -4 & 0 & -4 & 2 & -2 & 6 & 2 & -4 & 0 \\
\hline
26 & 8 & -2 & -2 & -4 & 0 & -2 & 2 & 0 & 0 & 2 & 2 & -4 & 0 & 2 & -2 \\
\hline
27 & -4 & 10 & 6 & 0 & 0 & 6 & -2 & 0 & -4 & -2 & 2 & -8 & 0 & 2 & -6 \\
\hline
28 & 0 & 4 & -4 & 4 & 0 & -4 & 0 & 8 & 0 & -4 & 4 & 4 & 0 & 4 & 0 \\
\hline
29 & 0 & 0 & 0 & 4 & 0 & 0 & -4 & 4 & 4 & -4 & 4 & 8 & -4 & 4 & 0 \\
\hline
30 & -4 & 0 & 4 & -2 & 2 & 2 & -2 & 0 & 4 & -4 & 0 & 2 & -2 & 2 & -2 \\
\hline
31 & 4 & -4 & 8 & 2 & -2 & -6 & -2 & -12 & 0 & 4 & 0 & 2 & 6 & -2 & 2 \\
\hline
32 & 0 & 0 & 0 & 0 & 0 & 0 & 0 & 0 & 0 & 0 & 0 & 0 & 0 & 0 & 0 \\
\hline
33 & 0 & 0 & 0 & 0 & 0 & 0 & 0 & 0 & 0 & 0 & 0 & 0 & 0 & 0 & 0 \\
\hline
34 & 0 & -4 & 4 & 6 & -6 & 2 & -2 & 4 & 4 & -4 & -12 & -2 & -6 & -2 & 2 \\
\hline
35 & 0 & -4 & -4 & 2 & 6 & -2 & 2 & 4 & -4 & -4 & 4 & -6 & -2 & -6 & -2 \\
\hline
36 & 0 & -2 & -2 & -2 & -6 & 4 & -8 & 0 & 0 & -2 & -2 & 2 & 6 & 8 & 4 \\
\hline
37 & -4 & -2 & 2 & -2 & -2 & 4 & 4 & -4 & 0 & 2 & -2 & -2 & -2 & 12 & -4 \\
\hline
38 & 4 & -10 & 2 & -12 & 0 & 2 & -2 & -4 & 0 & -2 & 10 & 0 & -4 & 2 & -2 \\
\hline
39 & 0 & 6 & -2 & 0 & 0 & -2 & -2 & 0 & 0 & -6 & 2 & 0 & 0 & -6 & -6 \\
\hline
40 & 0 & 4 & -4 & 2 & -2 & -2 & -6 & 0 & -4 & 4 & 8 & 6 & -2 & -6 & 2 
    \end{tabular}
    
    \begin{tabular}{c||c|c|c|c|c|c|c|c|c|c|c|c|c|c|c|c}
        & 1 & 2 & 3 & 4 & 5 & 6 & 7 & 8 & 9 & 10 & 11 & 12 & 13 & 14 & 15 \\
        \hline
        \hline
41 & 0 & 0 & -8 & 2 & -2 & 2 & -2 & -12 & 0 & -4 & 0 & 2 & -6 & 2 & -6 \\
\hline
42 & 4 & 0 & -4 & 4 & -8 & -4 & 0 & -4 & 4 & 0 & 0 & 0 & 0 & -4 & -4 \\
\hline
43 & 4 & -4 & 0 & 8 & 12 & 4 & 0 & -8 & 8 & 0 & 0 & -8 & 0 & 0 & 0 \\
\hline
44 & -4 & 2 & 6 & 0 & -4 & -6 & -2 & 0 & -8 & 2 & 2 & 0 & 0 & 2 & 10 \\
\hline
45 & 0 & -2 & -2 & 0 & -8 & -2 & -10 & 0 & 4 & -2 & 2 & -8 & -4 & -2 & 2 \\
\hline
46 & -4 & -6 & -2 & 10 & 2 & -4 & -4 & -4 & -4 & -6 & 2 & -6 & 6 & 0 & 4 \\
\hline
47 & 0 & 6 & -2 & -2 & 2 & 4 & 0 & -4 & 0 & 6 & 2 & 6 & -2 & 0 & 0 \\
\hline
        48 & 4 & -2 & 2 & -4 & -4 & 6 & 6 & 0 & -8 & -2 & -2 & -4 & 0 & -10 & 2 \\
        \hline
49 & 0 & -2 & -2 & 4 & -8 & -2 & 2 & -4 & 0 & 2 & -2 & 0 & -8 & 2 & 2 \\
\hline
50 & 0 & 2 & 2 & 2 & 2 & 0 & 0 & -4 & 0 & 2 & -2 & 2 & -2 & -4 & 0 \\
\hline
51 & 4 & 2 & -2 & -2 & -6 & -4 & 0 & 0 & 0 & -2 & -2 & 2 & 2 & 4 & 4 \\
\hline
52 & -8 & 0 & -8 & 6 & 6 & -6 & 2 & 4 & 0 & 4 & -8 & -2 & -6 & 2 & -2 \\
\hline
53 & -8 & 0 & 0 & -2 & -2 & 2 & 2 & -4 & 0 & -4 & 0 & -2 & 2 & 2 & -2 \\
\hline
54 & 0 & 0 & 0 & -4 & 0 & 0 & -4 & 0 & 4 & 4 & 0 & -4 & 4 & -4 & 4 \\
\hline
55 & 8 & 0 & -8 & -8 & 4 & -4 & 0 & 8 & 4 & -4 & 0 & 0 & 0 & 0 & 0 \\
\hline
56 & 0 & 2 & 2 & -6 & 2 & 0 & 0 & 4 & -4 & -2 & -2 & -6 & -6 & 0 & 0 \\
\hline
57 & 4 & 6 & -6 & 2 & 6 & 4 & 0 & -4 & -8 & 10 & -2 & 2 & -2 & 4 & 0 \\
\hline
58 & 0 & -2 & 6 & 4 & 0 & -10 & 2 & 0 & 0 & 2 & 2 & -4 & 0 & 2 & -2 \\
\hline
59 & -4 & 2 & -2 & -8 & 8 & -10 & -2 & -8 & -4 & -2 & -6 & 0 & 0 & 2 & 2 \\
\hline
60 & 0 & -4 & -4 & 4 & 0 & 4 & 0 & 0 & -8 & 4 & 4 & -4 & -8 & -4 & 0 \\
\hline
61 & 0 & 0 & 8 & -4 & 8 & -8 & -4 & 4 & 4 & 4 & 4 & 0 & 4 & 4 & -8 \\
\hline
62 & 4 & -12 & 0 & 6 & 2 & -2 & 2 & 4 & -8 & -4 & 0 & 6 & 2 & 2 & -2 \\
\hline
63 & -4 & -8 & -4 & -6 & -2 & -2 & -6 & 0 & -4 & 4 & 0 & -2 & 2 & -2 & 2
\end{tabular}

\medskip

Наибольшее по модулю число в этой таблице находится на позиции (18, 13), оно равно -14. Тогда уравнение
$$\vec{X}[2, 5] \oplus \vec{Y}[1, 2, 4] = \vec{K}[2, 5]$$
является эффективным линейным статистическим аналогом 2-го S-бокса в классе всех линейных статистических аналогов вида
$$(\vec{Y}, \vec{j}) = (\vec{X} \oplus \vec{K}, \vec{i})$$
с вероятностью $p_2 = \frac{-14 + 32}{64} = \frac{9}{32}$ и $\Delta_2 = |1 - \frac{9}{16}| = \frac{7}{16}$.

Получаем $\Delta_1 > \Delta_2$, значит, эффективным линейным статистическим аналогом произвольного раунда DES является уравнение:

$$\vec{X}_i [2] \oplus \vec{Y}_i [1, 2, 3, 4] = \vec{K}_i [2]$$.

С учётом расширения и перестановки:
$$\vec{X}_i [3] \oplus \vec{Y}_i [8, 1, 5, 3] = \vec{K}_i [2]$$

Запишем для первого и третьего раунда:
$$\vec{P}_L [3] \oplus (\vec{X}_2 \oplus \vec{P}_H)  [8, 1, 5, 3] = \vec{K}_1 [2]$$
$$(\vec{C}_L \oplus \vec{Y}_4) [3] \oplus (\vec{X}_2 \oplus \vec{C}_L) [8, 1, 5, 3] = \vec{K}_3 [2]$$

Ещё нам понадобится уравнение, содержащее $\vec{Y}_4[3]$. До перестановки это четвёртый бит первого S-бокса. То есть, надо искать по первому столбцу. На позиции (63, 1) максимальный по модулю элемент -12. Тогда $p_* = \frac{-12 + 32}{64} = \frac{5}{16}$ и $\Delta_* = |1 - \frac{5}{8}| = \frac{3}{8}$

$$\vec{X}_i [1, 2, 3, 4, 5, 6] \oplus \vec{Y}_i [4] = \vec{K}_i [1, 2, 3, 4, 5, 6]$$

Учтя расширение и перестановку, получим:
$$\vec{X}_i [1, 2] \oplus \vec{Y}_i [3] = \vec{K}_i [1, 2, 3, 4, 5, 6]$$

Запишем для четвёртого раунда
$$\vec{C}_L [1, 2] \oplus \vec{Y}_4 [3] = \vec{K}_4 [1, 2, 3, 4, 5, 6]$$

Сложив все уравнения, получим:

$$\vec{P}_L [3] \oplus \vec{P}_H [1, 3, 5, 8] \oplus \vec{C}_L [2, 5, 8]= \vec{K}_1 [2] \oplus \vec{K}_3 [2] \oplus \vec{K}_4 [1, 2, 3, 4, 5, 6]$$

Получим результирующую эффективность и вероятность:
$$\Delta = \Delta_1 \cdot \Delta_1 \cdot \Delta_* = \frac{9}{16} \cdot \frac{9}{16} \cdot \frac{3}{8} = \frac{243}{2048} \approx 0.12$$
$$p = \frac{1 - \Delta}{2} = \frac{1805}{4096} \approx 0.44 $$

Тогда можно раскрыть 8 бит ключа $K = (K_1, K_2, K_3, K_4)$, зная \\$|p - \frac{1}{2}|^{-2} = 284$ открытых текста с вероятностью успеха $97.7\%$. Из статьи Мицуру Мацуи можно сделать вывод, что лучший стат. аналог для оригинального $DES/4$ требует $269$ открытых текстов для раскрытия 2 бит ключа. То есть, стало хуже примерно в 4 раза.

Это означает, что "мудрить" с алгоритмами нельзя, а подбирать все параметры нужно крайне обдуманно, иначе можно значительно ухудшить стойкость криптографических алгоритмов.