\section{Шифры перестановки.}

\task{Раскрыть шифр простой замены:

\noindent 56 73 31 68 52 88 52 70 16 78 16 90 40 49 16 31 78 56 46 28 88 31 40 88 70 68 52 40 19 56 70 73 88 19 94 00 52 31 49 68 78 88 56 90 73 16 31 49 94 88 88 46 36 49 88 52 88 46 68 74 49 16 78 64 94 88 52 40 68 19 94 16 03 20 49 64 46 88 78 64 13 16 90 40 49 03 16 52 31 78 16 70 88 73 68 78 88 90 40 49 20 94 56 66 46 00 88 49 40 68 78 88 73 31 74 87 88 16 83 16 78 68 94 56 16 16 52 20 90 68 73 56 70 88 73 68 49 64 49 03 87 56 94 16 73 16 31 16 78 56 78 56 31 64 46 00 88 94 56 40 88 40 88 73 88 70 20 16 28 88 73 16 03 94 00 66 94 16 70 88 19 68 90 20 52 16 94 56 82 31 83 16 94 11 56 94 68 52 56 90 40 49 90 94 68 74 90 40 49 03 49 88 31 78 68 73 88 82 70 68 52 31 87 88 28 88 20 28 88 70 94 56 87 68 83 68 87 88 46 74 90 68 94 46 88 74 90 94 56 31 40 68 49 64 73 88 70 56 94 88 03 16 31 49 73 16 90 40 49 68 94 16 40 19 56 19 88 70 94 88 82 88 90 68 46 88 03 16 94 94 88 31 49 56 49 03 87 68 31 94 16 70 68 73 94 56 66 40 88 19 13 20 49 56 73 88 73 31 16 31 49 19 68 13 56 78 31 74 90 68 31 00 40 68 49 64 56 90 90 68 40 88 31 49 88 74 94 94 00 66 87 88 13 52 68 19 88 73 49 03 87 66 88 19 88 13 88 16 11 16 90 40 49 03 49 88 88 94 40 88 94 68 49 20 19 16 03 16 78 88 73 16 87 78 16 28 87 88 52 00 31 78 16 94 94 00 82 56 94 16 31 40 88 31 88 46 94 00 82 90 68 97 56 87 78 56 73 68 49 64 31 74 94 68 03 16 52 78 56 46 88 90 40 49 56 94 68 03 16 52 88 83 94 88 56 20 52 88 52 49 19 88 94 20 49 64 31 74 
}

Для более простого воспроизведения описанных действий буду приводить код на языке Python.

Проанализируем частоты монограмм.

\begin{lstlisting}
>>> sorted(zip(*np.unique(cipher, return_counts = True)), key = lambda x: x[1], reverse = True)[:10]
[('88', 58), ('16', 37), ('94', 36), ('68', 33), ('49', 31), ('56', 29), ('31', 26), ('40', 21), ('73', 19), ('90', 19)]
\end{lstlisting}


Теперь рассмотрим биграммы:

\begin{lstlisting}
>>> bigram = np.array([cipher[i] + ' ' + cipher[i+1] for i in range(len(cipher) - 1) ])
>>> sorted(zip(*np.unique(bigram, return_counts = True)), key = lambda x: x[1], reverse = True)[:10]
[('40 49', 8), ('88 73', 8), ('90 40', 8), ('40 88', 7), ('94 56', 7), ('03 16', 6), ('16 31', 6), ('31 49', 6), ('49 03', 6), ('49 64', 6)]
\end{lstlisting}

Наиболее частые моно- и биграммы русского языка:

\medskip

{\centering
\begin{tabular}{||c|c|c|c|c|c|c|c|c|c||}
\hline
\textbf{О} & \textbf{Е} & \textbf{А} & \textbf{И} & \textbf{Н} & \textbf{Т} & \textbf{С} & \textbf{Р} & \textbf{В} & \textbf{Л} \\
\hline
\end{tabular}

}

\medskip

{\centering
\begin{tabular}{||c|c|c|c|c|c|c|c|c|c||}
\hline
\textbf{СТ} & \textbf{НО} & \textbf{ЕН} & \textbf{ТО} & \textbf{НА} & \textbf{ОВ} & \textbf{НИ} & \textbf{РА} & \textbf{ВО} & \textbf{КО} \\
\hline
\end{tabular}

}

\medskip


\medskip

Предположим, что 88 -- это О. В биграммах из текста эта буква встречается дважды: 88 73 и 40 88. В справочной таблице единственное сочетание, в котором О стоит на первом месте -- это ОВ. Сравнивая позицию буквы 73 с первой таблицей, можем убедиться, что В действительно подходит. 

Допустим также, что 16 -- это Е. Поскольку в шифротексте нет явных знаков препинания, предположим, что они записаны в виде ЗПТ и ТЧК. Запятых, скорее всего, больше, чем точек, поэтому рассмотрим триграммы текста и самую частую определим как ЗПТ.

\begin{lstlisting}
>>> trigram = np.array([cipher[i] + ' ' + cipher[i+1] + ' ' + cipher[i+2] for i in range(len(cipher) - 2) ])
>>> sorted(zip(*np.unique(trigram, return_counts = True)), key = lambda x: x[1], reverse = True)[:5]
[('90 40 49', 8), ('16 90 40', 4), ('68 49 64', 4), ('03 16 52', 3), ('16 31 49', 3)]
\end{lstlisting}

Тогда 49 -- это Т. Попробуем найти среди биграмм наиболее частую -- СТ: единственный вариант, заканчивающийся на 49, -- это 31 49 (40 49 уже занято -- ПТ). Пусть 31 будет С.

Итак, попробуем подставить:

\medskip

{\centering
\begin{tabular}{||c|c|c|c|c|c|c||}
\hline
\textbf{О} & \textbf{В} & \textbf{Е} & \textbf{З} & \textbf{П} & \textbf{Т} & \textbf{С} \\
\hline
88 & 73 & 16 & 90 & 40 & 49 & 31  \\
\hline
\end{tabular}

}

\medskip

\begin{lstlisting}
>>> key = {'88': 'О', '73': 'В', '16': 'Е', '90': 'З', '40': 'П', '49': 'Т', '31': 'С'}
>>> ' '.join([key[x] if x in key else x for x in cipher])
'56 В С 68 52 О 52 70 Е 78 Е З П Т Е С 78 56 46 28 О С П О 70 68 52 П 19 56 70 В О 19 94 00 52 С Т 68 78 О 56 З В Е С Т 94 О О 46 36 Т О 52 О 46 68 74 Т Е 78 64 94 О 52 П 68 19 94 Е 03 20 Т 64 46 О 78 64 13 Е З П Т 03 Е 52 С 78 Е 70 О В 68 78 О З П Т 20 94 56 66 46 00 О Т П 68 78 О В С 74 87 О Е 83 Е 78 68 94 56 Е Е 52 20 З 68 В 56 70 О В 68 Т 64 Т 03 87 56 94 Е В Е С Е 78 56 78 56 С 64 46 00 О 94 56 П О П О В О 70 20 Е 28 О В Е 03 94 00 66 94 Е 70 О 19 68 З 20 52 Е 94 56 82 С 83 Е 94 11 56 94 68 52 56 З П Т З 94 68 74 З П Т 03 Т О С 78 68 В О 82 70 68 52 С 87 О 28 О 20 28 О 70 94 56 87 68 83 68 87 О 46 74 З 68 94 46 О 74 З 94 56 С П 68 Т 64 В О 70 56 94 О 03 Е С Т В Е З П Т 68 94 Е П 19 56 19 О 70 94 О 82 О З 68 46 О 03 Е 94 94 О С Т 56 Т 03 87 68 С 94 Е 70 68 В 94 56 66 П О 19 13 20 Т 56 В О В С Е С Т 19 68 13 56 78 С 74 З 68 С 00 П 68 Т 64 56 З З 68 П О С Т О 74 94 94 00 66 87 О 13 52 68 19 О В Т 03 87 66 О 19 О 13 О Е 11 Е З П Т 03 Т О О 94 П О 94 68 Т 20 19 Е 03 Е 78 О В Е 87 78 Е 28 87 О 52 00 С 78 Е 94 94 00 82 56 94 Е С П О С О 46 94 00 82 З 68 97 56 87 78 56 В 68 Т 64 С 74 94 68 03 Е 52 78 56 46 О З П Т 56 94 68 03 Е 52 О 83 94 О 56 20 52 О 52 Т 19 О 94 20 Т 64 С 74'
\end{lstlisting}

Обратим внимание на 'ЗПТЕС 78 56', 'ПОПОВО 70 20', 'ПОСТО 74 94 94 *', 'СПОСО 46'. Всё это похоже на ', если', 'по поводу', 'постоянн*' и 'способ'. Попробуем добавить в ключ следующие замены:

\medskip

{\centering
\begin{tabular}{||c|c|c|c|c|c|c||}
\hline
\textbf{Л} & \textbf{И} & \textbf{Д} & \textbf{У} & \textbf{Я} & \textbf{Н} & \textbf{Б} \\
\hline
78 & 56 & 70 & 20 & 74 & 94 & 46  \\
\hline
\end{tabular}

}

\medskip

\begin{lstlisting}
>>> key.update(**{'78': 'Л', '56': 'И','70': 'Д','20': 'У','74': 'Я','94': 'Н','46': 'Б'})
>>> ' '.join([key[x] if x in key else x for x in cipher])
'И В С 68 52 О 52 Д Е Л Е З П Т Е С Л И Б 28 О С П О Д 68 52 П 19 И Д В О 19 Н 00 52 С Т 68 Л О И З В Е С Т Н О О Б 36 Т О 52 О Б 68 Я Т Е Л 64 Н О 52 П 68 19 Н Е 03 У Т 64 Б О Л 64 13 Е З П Т 03 Е 52 С Л Е Д О В 68 Л О З П Т У Н И 66 Б 00 О Т П 68 Л О В С Я 87 О Е 83 Е Л 68 Н И Е Е 52 У З 68 В И Д О В 68 Т 64 Т 03 87 И Н Е В Е С Е Л И Л И С 64 Б 00 О Н И П О П О В О Д У Е 28 О В Е 03 Н 00 66 Н Е Д О 19 68 З У 52 Е Н И 82 С 83 Е Н 11 И Н 68 52 И З П Т З Н 68 Я З П Т 03 Т О С Л 68 В О 82 Д 68 52 С 87 О 28 О У 28 О Д Н И 87 68 83 68 87 О Б Я З 68 Н Б О Я З Н И С П 68 Т 64 В О Д И Н О 03 Е С Т В Е З П Т 68 Н Е П 19 И 19 О Д Н О 82 О З 68 Б О 03 Е Н Н О С Т И Т 03 87 68 С Н Е Д 68 В Н И 66 П О 19 13 У Т И В О В С Е С Т 19 68 13 И Л С Я З 68 С 00 П 68 Т 64 И З З 68 П О С Т О Я Н Н 00 66 87 О 13 52 68 19 О В Т 03 87 66 О 19 О 13 О Е 11 Е З П Т 03 Т О О Н П О Н 68 Т У 19 Е 03 Е Л О В Е 87 Л Е 28 87 О 52 00 С Л Е Н Н 00 82 И Н Е С П О С О Б Н 00 82 З 68 97 И 87 Л И В 68 Т 64 С Я Н 68 03 Е 52 Л И Б О З П Т И Н 68 03 Е 52 О 83 Н О И У 52 О 52 Т 19 О Н У Т 64 С Я'
\end{lstlisting}

Видно, что 'С Т 68 Л О И З В Е С Т Н О О Б 36 Т О 52 О Б 68 Я Т Е Л 64 Н О 52 П 68 19 Н Е' похоже на 'стало известно об этом обаятельном парне', а 'В Е С Е Л И Л И С 64 Б 00 О Н И П О П О В О Д У Е 28 О' -- на 'веселились бы они по поводу его', 'В О Д И Н О 03 Е С Т В Е' -- 'в одиночестве'

\medskip

{\centering
\begin{tabular}{||c|c|c|c|c|c|c|c||}
\hline
\textbf{А} & \textbf{Э} & \textbf{М} & \textbf{Ь} & \textbf{Р} & \textbf{Ы} & \textbf{Г} & \textbf{Ч} \\
\hline
68 & 36 & 52 & 64 & 19 & 00 & 28 & 03 \\
\hline
\end{tabular}

}
\medskip

\begin{lstlisting}
>>> key.update(**{'68': 'А', '36': 'Э','52': 'М','64': 'Ь','19': 'Р','00': 'Ы','28': 'Г', '03': 'Ч'})
>>> ' '.join([key[x] if x in key else x for x in cipher])
'И В С А М О М Д Е Л Е З П Т Е С Л И Б Г О С П О Д А М П Р И Д В О Р Н Ы М С Т А Л О И З В Е С Т Н О О Б Э Т О М О Б А Я Т Е Л Ь Н О М П А Р Н Е Ч У Т Ь Б О Л Ь 13 Е З П Т Ч Е М С Л Е Д О В А Л О З П Т У Н И 66 Б Ы О Т П А Л О В С Я 87 О Е 83 Е Л А Н И Е Е М У З А В И Д О В А Т Ь Т Ч 87 И Н Е В Е С Е Л И Л И С Ь Б Ы О Н И П О П О В О Д У Е Г О В Е Ч Н Ы 66 Н Е Д О Р А З У М Е Н И 82 С 83 Е Н 11 И Н А М И З П Т З Н А Я З П Т Ч Т О С Л А В О 82 Д А М С 87 О Г О У Г О Д Н И 87 А 83 А 87 О Б Я З А Н Б О Я З Н И С П А Т Ь В О Д И Н О Ч Е С Т В Е З П Т А Н Е П Р И Р О Д Н О 82 О З А Б О Ч Е Н Н О С Т И Т Ч 87 А С Н Е Д А В Н И 66 П О Р 13 У Т И В О В С Е С Т Р А 13 И Л С Я З А С Ы П А Т Ь И З З А П О С Т О Я Н Н Ы 66 87 О 13 М А Р О В Т Ч 87 66 О Р О 13 О Е 11 Е З П Т Ч Т О О Н П О Н А Т У Р Е Ч Е Л О В Е 87 Л Е Г 87 О М Ы С Л Е Н Н Ы 82 И Н Е С П О С О Б Н Ы 82 З А 97 И 87 Л И В А Т Ь С Я Н А Ч Е М Л И Б О З П Т И Н А Ч Е М О 83 Н О И У М О М Т Р О Н У Т Ь С Я'
\end{lstlisting}

'ЧУТЬБОЛЬ 13 ЕЗПТ' -- 'чуть больше,', 'УНИ 66 БЫ' -- 'у них бы', 'В С Я 87 О Е 83 Е Л А Н И Е' -- 'всякое желание', 'Н Е Д О Р А З У М Е Н И 82 С 83 Е Н 11 И Н А М И' -- 'недоразумений с женщинами', 'З А 97 И 87 Л И В А Т Ь С Я' -- 'зацикливаться'.

\medskip

{\centering
\begin{tabular}{||c|c|c|c|c|c|c||}
\hline
\textbf{Ш} & \textbf{Х} & \textbf{К} & \textbf{Ж} & \textbf{Й} & \textbf{Щ} & \textbf{Ц} \\
\hline
13 & 66 & 87 & 83 & 82 & 11 & 97 \\
\hline
\end{tabular}

}
\medskip

\begin{lstlisting}
>>> key.update(**{'13': 'Ш', '66': 'Х','87': 'К','83': 'Ж','82': 'Й','11': 'Щ','97': 'Ц'})
>>> ' '.join([key[x] if x in key else x for x in cipher])
'И В С А М О М Д Е Л Е З П Т Е С Л И Б Г О С П О Д А М П Р И Д В О Р Н Ы М С Т А Л О И З В Е С Т Н О О Б Э Т О М О Б А Я Т Е Л Ь Н О М П А Р Н Е Ч У Т Ь Б О Л Ь Ш Е З П Т Ч Е М С Л Е Д О В А Л О З П Т У Н И Х Б Ы О Т П А Л О В С Я К О Е Ж Е Л А Н И Е Е М У З А В И Д О В А Т Ь Т Ч К И Н Е В Е С Е Л И Л И С Ь Б Ы О Н И П О П О В О Д У Е Г О В Е Ч Н Ы Х Н Е Д О Р А З У М Е Н И Й С Ж Е Н Щ И Н А М И З П Т З Н А Я З П Т Ч Т О С Л А В О Й Д А М С К О Г О У Г О Д Н И К А Ж А К О Б Я З А Н Б О Я З Н И С П А Т Ь В О Д И Н О Ч Е С Т В Е З П Т А Н Е П Р И Р О Д Н О Й О З А Б О Ч Е Н Н О С Т И Т Ч К А С Н Е Д А В Н И Х П О Р Ш У Т И В О В С Е С Т Р А Ш И Л С Я З А С Ы П А Т Ь И З З А П О С Т О Я Н Н Ы Х К О Ш М А Р О В Т Ч К Х О Р О Ш О Е Щ Е З П Т Ч Т О О Н П О Н А Т У Р Е Ч Е Л О В Е К Л Е Г К О М Ы С Л Е Н Н Ы Й И Н Е С П О С О Б Н Ы Й З А Ц И К Л И В А Т Ь С Я Н А Ч Е М Л И Б О З П Т И Н А Ч Е М О Ж Н О И У М О М Т Р О Н У Т Ь С Я'
>>> key
{'88': 'О', '73': 'В', '16': 'Е', '90': 'З', '40': 'П', '49': 'Т', '31': 'С', '78': 'Л', '56': 'И', '70': 'Д', '20': 'У', '74': 'Я', '94': 'Н', '46': 'Б', '68': 'А', '36': 'Э', '52': 'М', '64': 'Ь', '19': 'Р', '00': 'Ы', '28': 'Г', '03': 'Ч', '13': 'Ш', '66': 'Х', '87': 'К', '83': 'Ж', '82': 'Й', '11': 'Щ', '97': 'Ц'}
\end{lstlisting}

\task{Раскрыть шифр вертикальной перестановки:

АЕЧСЕ ЛЫЯИЛ ОПЗИЕ СТЫБД ТТДРД ОВИГР ЙВКАЛ МАШЛУ ПЗЖТЯ РОСЗГ ЕНОПЫ ИОМЕО ОЯТТХ ОДАЛР УИВИО ООННИ ОВЫЫБ ИАОРС ОТГАБ СОЕЧД ВУНЛУ НИМОЕ ШШАВН ЕАВМЙ
}

Длина текста 120 букв. Наиболее целесообразно было бы использовать ключ длины 10 или 12 (близкой к $\sqrt{120}$). Проверим различные длины ключей на основе известного соотношения гласных к согласным: 44\% к 56\%.

\medskip

\begin{lstlisting}
>>> def get_mse(text, n):
...     vn = lambda row: sum([x in list('АЕЁИОУЫЭЮЯ') for x in row])
...     table = np.array(list(text)).reshape((n, len(text) // n)).T
...     ratio =  np.array([vn(row) / len(row) for row in table])
... 	return sum((ratio - 0.44) ** 2) / (len(text) // n)
...
>>> mse = [(round(get_mse(text, i), 5), i) for i in [6, 8, 10, 12, 15]]
>>> sorted(mse, key = lambda x: x[0])
[(0.00216, 15), (0.02229, 12), (0.02577, 10), (0.03514, 8), (0.03966, 6)]
\end{lstlisting}

Видим, что наименьшая среднеквадратичная ошибка достигается при ключе длины 15.

{\centering
\begin{tabular}{||c|c|c|c|c|c|c|c|c|c|c|c|c|c|c||}
\hline
1 & 2 & 3 & 4 & 5 & 6 & 7 & 8 & 9 & 10 & 11 & 12 & 13 & 14 & 15 \\
\hline
А & И & Т & Д & К & П & З & О & Х & В & О & Р & О & У & А \\
\hline
Е & Л & Ы & О & А & З & Г & М & О & И & В & С & Е & Н & В \\
\hline
Ч & О & Б & В & Л & Ж & Е & Е & Д & О & Ы & О & Ч & И & Н \\
\hline
С & П & Д & И & М & Т & Н & О & А & О & Ы & Т & Д & М & Е \\
\hline
Е & З & Т & Г & А & Я & О & О & Л & О & Б & Г & В & О & А \\
\hline
Л & И & Т & Р & Ш & Р & П & Я & Р & Н & И & А & У & Е & В \\
\hline
Ы & Е & Д & Й & Л & О & Ы & Т & У & Н & А & Б & Н & Ш & М \\
\hline
Я & С & Р & В & У & С & И & Т & И & И & О & С & Л & Ш & Й \\
\hline
\end{tabular}

}
\medskip

Обратим внимание на столбцы, в которых есть буква 'Ы' -- с ними будет проще всего найти невстречающиеся биграммы. Например, столбец 11 сочетается только с 3 и 5 столбцами. Так как, например, 'ДЫМ' встретится чаще, чем 'МЫД', поставим столбцы в порядке 3 - 11 - 5. Во второй строке получаем триграмму 'ЫВА', после которой может быть 'Н', 'Т', 'Е', 'Ю', 'Л', 'Я'. Отметим кандидатами 1, 2, 13 и 14 столбец. В последней строке получается 'РОУЯ', если выбрать первый столбец -- отбраковываем, при 14-м столбце в 5-й строке получится 'ТБАО' -- отбраковываем. На третьей строке скорее будет 'БЫЛО', чем 'БЫЛЧ', поэтому остановимся на варианте 3 - 11 - 5 - 2.

\medskip

{\centering
\begin{tabular}{||c|c|c|c|c|c|c|c|c|c|c|c|c|c|c||}
\hline
1 & 6 & 3 & 11 & 5 & 2 & 7 & 8 & 9 & 10 & 4 & 12 & 13 & 14 & 15 \\
\hline
А & П & \bf Т & \bf О & \bf К & \bf И & З & О & Х & В & Д & Р & О & У & А \\
\hline
Е & З & \bf Ы & \bf В & \bf А & \bf Л & Г & М & О & И & О & С & Е & Н & В \\
\hline
Ч & Ж & \bf Б & \bf Ы & \bf Л & \bf О & Е & Е & Д & О & В & О & Ч & И & Н \\
\hline
С & Т & \bf Д & \bf Ы & \bf М & \bf П & Н & О & А & О & И & Т & Д & М & Е \\
\hline
Е & Я & \bf Т & \bf Б & \bf А & \bf З & О & О & Л & О & Г & Г & В & О & А \\
\hline
Л & Р & \bf Т & \bf И & \bf Ш & \bf И & П & Я & Р & Н & Р & А & У & Е & В \\
\hline
Ы & О & \bf Д & \bf А & \bf Л & \bf Е & Ы & Т & У & Н & Й & Б & Н & Ш & М \\
\hline
Я & С & \bf Р & \bf О & \bf У & \bf С & И & Т & И & И & В & С & Л & Ш & Й \\
\hline
\end{tabular}
}

\medskip

В первой строке видно слово 'ВОЗДУХ', 10 - (8, 13) - 7 - 4 - 14 - 9. На третьей строке оказывается 'ОЕЕ', если выбрать 8-й столбец, и 'ОЧЕ', если выбрать 13-й. Установим столбцы по второму варианту.

\medskip

{\centering
\begin{tabular}{||c|c|c|c|c|c|c|c|c|c|c|c|c|c|c||}
\hline
1 & 6 & 3 & 11 & 5 & 2 & 12 & 8 & 15 & 10 & 13 & 7 & 4 & 14 & 9 \\
\hline
А & П & \bf Т & \bf О & \bf К & \bf И & Р & О & А & \bf В & \bf О & \bf З & \bf Д & \bf У & \bf Х \\
\hline
Е & З & \bf Ы & \bf В & \bf А & \bf Л & С & М & В & \bf И & \bf Е & \bf Г & \bf О & \bf Н & \bf О \\
\hline
Ч & Ж & \bf Б & \bf Ы & \bf Л & \bf О & О & Е & Н & \bf О & \bf Ч & \bf Е & \bf В & \bf И & \bf Д \\
\hline
С & Т & \bf Д & \bf Ы & \bf М & \bf П & Т & О & Е & \bf О & \bf Д & \bf Н & \bf И & \bf М & \bf А \\
\hline
Е & Я & \bf Т & \bf Б & \bf А & \bf З & Г & О & А & \bf О & \bf В & \bf О & \bf Г & \bf О & \bf Л \\
\hline
Л & Р & \bf Т & \bf И & \bf Ш & \bf И & А & Я & В & \bf Н & \bf У & \bf П & \bf Р & \bf Е & \bf Р \\
\hline
Ы & О & \bf Д & \bf А & \bf Л & \bf Е & Б & Т & М & \bf Н & \bf Н & \bf Ы & \bf Й & \bf Ш & \bf У \\
\hline
Я & С & \bf Р & \bf О & \bf У & \bf С & С & Т & Й & \bf И & \bf Л & \bf И & \bf В & \bf Ш & \bf И \\
\hline
\end{tabular}
}
\medskip

Видно, что эти два блока можно объединить. Кроме того, можно заметить слова 'ПОТОКИ' и 'ОЧЕВИДНО': 9 - 15 - 12, 6 - 8 - 3. Остаётся последний столбец, для которого становится ясно, что он должен находиться в конце таблицы.

Окончательный ответ:

\medskip

{\centering\bf
\begin{tabular}{||c|c|c|c|c|c|c|c|c|c|c|c|c|c|c||}
\hline
П & О & Т & О & К & И & В & О & З & Д & У & Х & А & Р & А \\
\hline
З & М & Ы & В & А & Л & И & Е & Г & О & Н & О & В & С & Е \\
\hline
Ж & Е & Б & Ы & Л & О & О & Ч & Е & В & И & Д & Н & О & Ч \\
\hline
Т & О & Д & Ы & М & П & О & Д & Н & И & М & А & Е & Т & С \\
\hline
Я & О & Т & Б & А & З & О & В & О & Г & О & Л & А & Г & Е \\
\hline
Р & Я & Т & И & Ш & И & Н & У & П & Р & Е & Р & В & А & Л \\
\hline
О & Т & Д & А & Л & Е & Н & Н & Ы & Й & Ш & У & М & Б & Ы \\
\hline
С & Т & Р & О & У & С & И & Л & И & В & Ш & И & Й & С & Я \\
\hline
\end{tabular}
}

\medskip