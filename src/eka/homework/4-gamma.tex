\section{Перекрытия гаммы. Криптоанализ при неравновероятной гамме.}

\task{Два текста $x$ и $x'$ на русском языке зашифрованы шифром гамирования по $\mod 30$ с помощью одной и той же гаммы $\gamma$. Использована следующая таблица соответствия букв числами (здесь -- означает пробел): \\

\medskip
{\centering
\begin{tabular}{||c|c|c|c|c|c|c|c|c|c|c|c|c|c|c||}
\hline
А & Б & В & Г & Д & Е & Ж & З & И & К & Л & М & Н & О & П \\
\hline
00 & 01 & 02 & 03 & 04 & 05 & 06 & 07 & 08 & 09 & 10 & 11 & 12 & 13 & 14\\
\hline
Р & С & Т & У & Ф & Х & Ц & Ч & Ш & Щ & Ы & Э & Ю & Я & -- \\
\hline
15 & 16 & 17 & 18 & 19 & 20 & 21 & 22 & 23 & 24 & 25 & 26 & 27 & 28 & 29 \\
\hline
\end{tabular}

}
\medskip

Получено два шифротекста $y =$ КЛОВБЛЖЗФ и $y'=$ ВУПЗЕРСВЖ, известна тематика $x$ и $x'$: 'времена года'. Применяя 'протяжку вероятного слова' найти $x, x', \gamma$.

}

Переведём векторы $y$ и $y'$ в числа и найдём их разность:

\noindent $y - y' = x + \gamma - x' - \gamma = x - x' = (9 - 2, 10 - 18, 13 - 14, 2 - 7, 1 - 5, 10 - 15, \\ 6 - 16, 7 - 2, 19 - 6) = (7, 22, 29, 25, 26, 25, 20, 5, 13) = $ ЗЧ--ЫЭЫЧЕО.

Попробуем подставить в начало $x'$ слово 'ЗИМА--':

\noindent $x = (x - x') + x' = АСНЕГ * * * *$

Видно, что получается осмысленное предложение. Посмотрим, какая гамма:

$\gamma = y' - x' = ВВВВВ ****$

Предположим, что гамма состоит только из этих букв, продлим и получим окончательный ответ:

\noindent $x = ЗИМА-ИДЕТ$

\noindent $x' = АСНЕГОПАД$

\noindent $\gamma = ВВВВВВВВВ$

\task{Пусть в шифре гаммирования по mod 30 используется только 6 знаков гаммы \(\{17, 05, 02, 15, 08, 14\} \) (соответствие букв и чисел в таблице):

\medskip
{\centering
\begin{tabular}{||c|c|c|c|c|c|c|c|c|c|c|c|c|c|c||}
\hline
А & Б & В & Г & Д & Е & Ж & З & И & К & Л & М & Н & О & П \\
\hline
00 & 01 & 02 & 03 & 04 & 05 & 06 & 07 & 08 & 09 & 10 & 11 & 12 & 13 & 14\\
\hline
Р & С & Т & У & Ф & Х & Ц & Ч & Ш & Щ & Ы & Ъ & Э & Ю & Я  \\
\hline
15 & 16 & 17 & 18 & 19 & 20 & 21 & 22 & 23 & 24 & 25 & 26 & 27 & 28 & 29 \\
\hline
\end{tabular}

}
\medskip


Получен шифртекст $y =$ ШАССЧАТАИЦОС. Используя "зигзагообразное" чтение дешифровать открытый текст и восстановить гамму. \\
}

Составим таблицу из возможных результатов гаммирования:

\medskip

{\centering
\begin{tabular}{||c|c|c|c|c|c|c|c|c|c|c|c|c|c|c||}
\hline
17 & Ж & О & Я & Я & Е & \bf О & А & О & Ц & Д & Ъ & \bf Я \\
\hline
05 & У & Ы & М & М & \bf Т & Ы & Н & Ы & Г & С & \bf И & М \\
\hline
02 & Ц & Ю & П & \bf П & Х & Ю & Р & Ю & Ж & \bf Ф & М & П \\
\hline
15 & И & \bf Р & Б & Б & З & Р & В & \bf Р & Ш & Ж & Ю & Б \\
\hline
08 & Р & Ч & \bf И & И & П & Ч & К & Ч & \bf А & О & Е & И \\
\hline
14 & \bf К & С & В & В & И & С & \bf Г & С & Щ & З & Я & В \\
\hline
\end{tabular}
}
\medskip

Легко видеть, $x = $ КРИПТОГРАФИЯ, $\gamma = $ ПРИВЕТПРИВЕТ.
